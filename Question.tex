\documentclass[12pt]{article}
% --- بسته‌های مورد نیاز ---
\usepackage{amsmath}         % برای محیط‌های پیشرفته ریاضی
\usepackage{amssymb}         % برای نمادهای ریاضی
\usepackage{enumitem}        % برای کنترل لیست‌ها
\usepackage[a4paper, margin=2.5cm]{geometry} % تنظیم حاشیه‌ها
\usepackage{tikz}
\usetikzlibrary{calc}
% --- تنظیمات نوشتار فارسی ---
\usepackage{hyperref} % حتما این بسته را اضافه کنید
\usetikzlibrary{positioning} % <--- این کتابخانه را برای موقعیت‌دهی نسبی اضافه کنید
\usetikzlibrary{intersections}
\usepackage{xepersian}
\settextfont{DibajFaNum-MediumContrastRegular}
\setdigitfont{DibajFaNum-MediumContrastRegular}
\renewcommand{\thesection}{\arabic{section}} % شماره‌گذاری بخش‌ها با اعداد
\begin{document}
	\begin{center}
		\Large\textbf{مجموعه سوالات پیشرفته ریاضی}
		\vspace{0.5cm}
	\end{center}
\begin{center}
	\Large\textbf{علیرضا کچویی} \\
	\vspace{0.5cm} % فاصله کمتر برای ایمیل
	\normalsize\texttt{Gmail:alirezakachouei1@gmail.com} % ایمیل شما در این خط
	\vspace{0.5cm}
\end{center}
	\hrulefill
	\vspace{1cm}
	% ------------------------------------------------------------------
	% سوال اول
	% ------------------------------------------------------------------
	\section*{سوال ۱: دستگاه معادلات مثلثاتی}
	دستگاه معادلات زیر را برای متغیرهای \(x\) و \(y\) حل کنید:
	\begin{displaymath}
		\begin{cases}
			\cos(x) - \cos(y) = \frac{1}{2} \\
			\sin(x) - \sin(y) = -\frac{\sqrt{3}}{2}
		\end{cases}
	\end{displaymath}
	
	
	\vspace{1cm}
	\hrulefill
	
	
	\vspace{1cm}
	% ------------------------------------------------------------------
	% سوال دوم
	% ------------------------------------------------------------------
	\section*{سوال ۲: معادله بر پایه‌ی نامساوی و کران‌داری}
	تمام جواب‌های حقیقی معادله‌ی زیر را بیابید:
	\begin{displaymath}
		\sin^8(x) + \cos^{14}(x) = 1
	\end{displaymath}
	
	
	\vspace{1cm}
	\hrulefill
	\vspace{1cm}
	% ------------------------------------------------------------------
	% سوال سوم
	% ------------------------------------------------------------------
	\section*{سوال ۳: معادله با اتحاد پنهان}
	معادله‌ی زیر را حل کنید:
	\begin{displaymath}
		\cos^3(x)\cos(3x) + \sin^3(x)\sin(3x) = \frac{1}{8}
	\end{displaymath}
	
	
	\vspace{1cm}
	\hrulefill
	\vspace{1cm}
	% ------------------------------------------------------------------
	% سوال چهارم
	% ------------------------------------------------------------------
	\section*{سوال ۴: معادله مثلثاتی خودآگاه}
	معادله \((n + 1)\cos(x) - n \sin(x) = n + 2\) را در نظر بگیرید. در این معادله، \(n\) برابر با تعداد ریشه‌های متمایز خود همین معادله در بازه \([0, 2\pi)\) است. مقدار \(n\) را بیابید.
	
	
	\vspace{1cm}
	\hrulefill
	\vspace{1cm}
	% ------------------------------------------------------------------
	% سوال پنجم
	% ------------------------------------------------------------------
	\section*{سوال ۵: ریشه‌های گمشده‌ی یک چندجمله‌ای}
	می‌دانیم که \( \tan(\frac{\pi}{12}) \) و \( \tan(\frac{5\pi}{12}) \) دو ریشه از سه ریشه‌ی معادله‌ی درجه سوم \( x^3 - 3x^2 - 3x + 1 = 0 \) هستند. مقدار \( \tan^2(\frac{\pi}{12}) + \tan^2(\frac{5\pi}{12}) \) کدام است؟
	
	
	\vspace{1cm}
	\hrulefill
	\vspace{1cm}
	% ------------------------------------------------------------------
	% سوال ششم
	% ------------------------------------------------------------------
	\section*{سوال ۶: هویت پنهان در توابع معکوس}
	فرض کنید \(A\) و \(B\) دو زاویه در بازه‌ی \( [-\frac{\pi}{2}, \frac{\pi}{2}] \) هستند به طوری که \(A+B = \frac{2\pi}{3}\). اگر \( \sin(A) = x \) و \( \sin(B) = y \) باشد، مقدار عبارت \( x\sqrt{1-y^2} + y\sqrt{1-x^2} \) چقدر است؟
	
	
	\vspace{1cm}
	\hrulefill
	\vspace{1cm}
	% ------------------------------------------------------------------
	% سوال هفتم
	% ------------------------------------------------------------------
	\section*{سوال ۷: ریشه‌های پنهان در چندجمله‌ای چبیشف}
	فرض کنید \(\sin(a) = x\). مجموع مربعات ریشه‌های معادله‌ی \( \sin(7a) = 0 \) در بازه‌ی \( (-1, 1) \) که از مقادیر \(x\) به دست می‌آیند، کدام است؟
	
	
	\vspace{1cm}
	\hrulefill
	\vspace{1cm}
	% ------------------------------------------------------------------
	% سوال هشتم
	% ------------------------------------------------------------------
	\section*{سوال ۸: دوره تناوب تابع جزء صحیح مثلثاتی}
	دوره تناوب اصلی (کوچکترین دوره تناوب مثبت) تابع \( f(x) = [2\sin(x)] + [2\cos(x)] \) چقدر است؟ (نماد \( [y] \) نشان‌دهنده‌ی جزء صحیح عدد \(y\) است.)
	
	
	\vspace{1cm}
	\vspace{1cm}
	% ------------------------------------------------------------------
	% سوال نهم
	% ------------------------------------------------------------------
	\section*{سوال ۹: دنباله بازگشتی مثلثاتی}
	دنباله‌ی \( (x_n) \) با رابطه‌ی بازگشتی \( x_{n+1} = 1 - 2x_n^2 \) و جمله‌ی اول \( x_0 \) تعریف شده است. چند مقدار اولیه متمایز برای \( x_0 \) در بازه \( [-1, 1] \) وجود دارد به طوری که \( x_{2025} = x_0 \) باشد؟
	
	
	
	\vspace{1cm}
	\hrulefill
	\vspace{1cm}
	% ------------------------------------------------------------------
	% سوال دهم
	% ------------------------------------------------------------------
	\section*{سوال ۱۰: معادله‌ی مثلثاتی با طعم براکت}
	تعداد جواب‌های معادله‌ی \( 2\tan(x) - [\tan(x)] = 0 \) در بازه‌ی \( [0, 2\pi) \) کدام است؟
	
	
	\vspace{1cm}
	\hrulefill
	\vspace{1cm}
	% ------------------------------------------------------------------
	% سوال یازدهم
	% ------------------------------------------------------------------
	\section*{سوال ۱۱: معادله براکتی }
	تعداد جواب‌های معادله‌ی \( [\tan(x)] = [\cot(x)] \) در بازه‌ی \( [0, 2\pi) \) کدام است؟
	
	
	\vspace{1cm}
	\hrulefill
	\vspace{1cm}
	% ------------------------------------------------------------------
	% سوال دوازدهم
	% ------------------------------------------------------------------
	\section*{سوال ۱۲: هویت پنهان در مجموع کسینوس‌ها}
	کوچکترین جواب مثبت معادله‌ی زیر کدام است؟
	\[ \cos(x) + \cos(2x) + \cos(3x) + \cos(4x) + \cos(5x) + \cos(6x) + \cos(7x) = -1 \]
	
	
	\vspace{1cm}
	\hrulefill
	\vspace{1cm}
	% ------------------------------------------------------------------
	% سوال سیزدهم
	% ------------------------------------------------------------------
	\section*{سوال ۱۳: معادله مثلثاتی با اتحادهای ترکیبی}
	معادله‌ی \( \sqrt{2} \tan(x) - \cos(x) = \sin(x) - \sqrt{2} \cot(x) \) در بازه‌ی \( [0, 2\pi] \) چند جواب دارد؟
	
	
	\vspace{1cm}
	\hrulefill
	\vspace{1cm}
	% ------------------------------------------------------------------
	% سوال چهاردهم
	% ------------------------------------------------------------------
	\section*{سوال ۱۴: معادله با هویت‌های نصف کمان}
	تعداد جواب‌های معادله‌ی \( \frac{1 - \cos(x)}{\sin(x)} + \frac{\sin(x)}{1 + \cos(x)} = 4\sin(\frac{x}{2}) \) در بازه‌ی \( [0, 2\pi] \) کدام است؟
	
	
	\vspace{1cm}
	\hrulefill
	\vspace{1cm}
	% ------------------------------------------------------------------
	% سوال پانزدهم
	% ------------------------------------------------------------------
	\section*{سوال ۱۵: معادله رادیکالی با دامنه پنهان}
	مجموع جواب‌های معادلهٔ مثلثاتی \( \sqrt{\frac{1-\cos(2x)}{\sin(x)}} - \sqrt{\frac{1+\cos(2x)}{\cos(x)}} = 0 \) در بازهٔ \( (0, 4\pi) \) کدام است؟
	
	
	\vspace{1cm}
	\hrulefill
	\vspace{1cm}
	% ------------------------------------------------------------------
	% سوال شانزدهم
	% ------------------------------------------------------------------
	\section*{سوال ۱۶: معادله قدر مطلقی با اتحادهای متقابل}
	تعداد جواب‌های معادله‌ی \( |\tan(x) - 1| + |\cot(x) - 1| = 2 \) در بازه‌ی \( [0, 2\pi) \) کدام است؟
	
	
	\vspace{1cm}
	\hrulefill
	\vspace{1cm}
	% ------------------------------------------------------------------
	% سوال هفدهم
	% ------------------------------------------------------------------
	\section*{سوال ۱۷: معادله تانژانت با آرگومان‌های ترکیبی}
	معادله‌ی \( \tan(2x) + \tan(x + \frac{\pi}{4}) = 1 \) در بازه‌ی \( [0, 2\pi] \) چند جواب دارد؟
	
	
	\vspace{1cm}
	\hrulefill
	\vspace{1cm}
	% ------------------------------------------------------------------
	% سوال هجدهم
	% ------------------------------------------------------------------
	\section*{سوال ۱۸: سیستم معادلات مثلثاتی پنهان}
	دستگاه معادلات زیر در بازه‌ی \( [0, 2\pi) \) چند زوج جواب \( (x, y) \) دارد؟
	\[ \begin{cases} \tan(x)\cot(y) = 3 \\ \cos(x)\sin(y) = \frac{1}{4} \end{cases} \]
	
	
	\vspace{1cm}
	\hrulefill
	\vspace{1cm}
	% ------------------------------------------------------------------
	% سوال نوزدهم
	% ------------------------------------------------------------------
	\section*{سوال ۱۹: معادله تانژانت با هویت جمع}
	مجموع جواب‌های معادلهٔ مثلثاتی \( \tan(2x) + \tan(x) = 1 - \tan(2x)\tan(x) \) در بازهٔ \( (0, \pi) \) کدام است؟
	
	
	\vspace{1cm}
	\hrulefill
	\vspace{1cm}
	% ------------------------------------------------------------------
	% سوال بیستم
	% ------------------------------------------------------------------
	\section*{سوال ۲۰: معادله با تغییر متغیر هوشمندانه}
	معادلهٔ \( \frac{\sin(x)+\cos(x)}{\sin(x)+\cos(x)+2} = 2\sin(2x)-3 \) در بازهٔ \( [0, 2\pi] \) چند جواب دارد؟
	
	
	\vspace{1cm}
	\hrulefill
	\vspace{1cm}
	% ------------------------------------------------------------------
	% سوال بیست و یکم
	% ------------------------------------------------------------------
	\section*{سوال ۲۱: تابع بازگشتی}
	تابع \( f(x) = \frac{x\sqrt{3} - 1}{x + \sqrt{3}} \) را در نظر بگیرید. همچنین دنباله‌ی توابع \( g_n(x) \) را به صورت \( g_1(x) = f(x) \) و \( g_{n+1}(x) = f(g_n(x)) \) تعریف می‌کنیم. مقدار \(x\) که در معادله‌ی \( g_{2024}(x) = 1 \) صدق می‌کند، کدام است؟


	\vspace{1cm}
	\hrulefill
	\vspace{1cm}
	% ------------------------------------------------------------------
	% سوال بیست و دوم
	% ------------------------------------------------------------------
	\section*{سوال ۲۲: معادله مثلثاتی با عبارت رادیکالی}
	
	تعداد جواب‌های معادلهٔ مثلثاتی \( \sin^2(x) + \sin(x) - \cos(x) = \tan(x)\cot(x) \) در بازهٔ \( (0, 2\pi) \) کدام است؟
	
	
	\vspace{1cm}
	\hrulefill
	\vspace{1cm}
	% ------------------------------------------------------------------
	% سوال بیست و سوم
	% ------------------------------------------------------------------
	\section*{سوال ۲۳: معادله رادیکالی با تحلیل علامت}
	تعداد جواب‌های معادله‌ی \( (\sin(x) - \cos(x)) \sqrt{\tan(x) + \cot(x) - 2} = \sin(2x) - 1 \) در بازه‌ی \( [0, 2\pi) \) کدام است؟
	
	\vspace{1cm}
	\hrulefill
	\vspace{1cm}
	% ------------------------------------------------------------------
	% سوال بیست و چهارم
	% ------------------------------------------------------------------
	\section*{سوال ۲۴: معادله مثلثاتی با هویت ترکیبی}
	فرض کنید \( \sin(x) + \cos(x) = a \). مقدار دقیق \( a \) را بیابید به طوری که عبارت زیر برقرار باشد:
	\[
	\sin^4(x) + \cos^4(x) + \sin^6(x) + \cos^6(x) = \frac{7}{8}
	\]
	\vspace{1cm}
	\hrulefill
	\vspace{1cm}
	% ------------------------------------------------------------------
	% سوال بیست و پنجم
	% ------------------------------------------------------------------
	\section*{سوال ۲۵: دستگاه معادلات لگاریتمی}
	دستگاه معادلات زیر را برای مقادیر \(x\) و \(y\) حل کنید:
	\begin{displaymath}
		\begin{cases}
			\log_{2}(x) + \log_{4}(y) = 4 \\
			\log_{x}(y) + \log_{y}(x) = \frac{5}{2}
		\end{cases}
	\end{displaymath}
	
	\vspace{1cm}
	\hrulefill
	\vspace{1cm}
	
	% ------------------------------------------------------------------
	% سوال بیست و ششم (سوال جدید)
	% ------------------------------------------------------------------
	\section*{سوال ۲۶: معادله‌ای در مرز منطق و جبر}
	دستگاه معادلات زیر را برای تمام مقادیر حقیقی \(x\) حل کنید:
	\begin{displaymath}
		\begin{cases}
			[\log_{2}x] \cdot \{\log_{2}x\} = \frac{1}{\sqrt{2}} \\
			\log_{4}x + \log_{x}4 = 3
		\end{cases}
	\end{displaymath}
	\paragraph{راهنمایی:}
	\begin{itemize}[rightmargin=2cm]
		\item نماد \( [y] \) نشان‌دهنده‌ی \textbf{جزء صحیح} عدد \(y\) است (بزرگترین عدد صحیحی که از \(y\) بزرگتر نیست).
		\item نماد \( \{y\} \) نشان‌دهنده‌ی \textbf{جزء کسری} عدد \(y\) است، که برابر است با \(y - [y]\).
	\end{itemize}
	
	\vspace{1cm}
	\hrulefill
	\vspace{1cm}
	
		% ------------------------------------------------------------------
	% سوال بیست و هفتم (سوال جدید)
	% ------------------------------------------------------------------
	\section*{سوال ۲۷: معادله‌ای با چهره‌ی مبدل}
	معادله‌ی زیر را برای تمام مقادیر حقیقی \(x\) حل کنید:
	\begin{displaymath}
		x^{3 - \log_{5}x} = 25
	\end{displaymath}
	
	\vspace{1cm}
	\hrulefill
	\vspace{1cm}
	
		% ------------------------------------------------------------------
	% سوال بیست و هشتم
	% ------------------------------------------------------------------
	\section*{سوال ۲۸: تحلیل پارامتری معادله گنگ}
	به ازای چند مقدار حقیقی برای پارامتر \(a\)، معادله‌ی زیر دقیقاً یک جواب برای \(x\) دارد؟
	\begin{displaymath}
		\frac{4}{x} - \frac{a}{x^2} = a + 2
	\end{displaymath}
	
	\vspace{1cm}
	\hrulefill
	\vspace{1cm}
	
		% ------------------------------------------------------------------
	% سوال بیست و نهم (سوال سخت‌تر)
	% ------------------------------------------------------------------
	\section*{سوال ۲۹: تحلیل پارامتری در معادله مثلثاتی}
	مجموع تمام مقادیر صحیح پارامتر \(a\) را بیابید که به ازای آن‌ها، معادله‌ی زیر دقیقاً سه جواب متمایز برای \(x\) در بازه‌ی \( [0, 2\pi) \) داشته باشد.
	\begin{displaymath}
		\cos(2x) + (2a-1)\sin(x) + a - 1 = 0
	\end{displaymath}
	
	\vspace{1cm}
	\hrulefill
	\vspace{1cm}
	
		% ------------------------------------------------------------------
	% سوال سی‌ام
	% ------------------------------------------------------------------
	\section*{سوال ۳۰: معادله با ساختار متقارن پنهان}
	معادله‌ی گویا زیر را برای تمام مقادیر حقیقی \(x\) حل کنید:
	\begin{displaymath}
		\frac{(x-1)^2 (x^2+1)}{(x^2-x+1)^2} = \frac{5}{9}
	\end{displaymath}
	
	\vspace{1cm}
	\hrulefill
	\vspace{1cm}
	
	
		% ------------------------------------------------------------------
	% سوال سی و یکم (سوال بسیار چالشی‌تر)
	% ------------------------------------------------------------------
	\section*{سوال ۳۱: معادله با تقارن پنهان دولایه}
	تمام ریشه‌های حقیقی معادله‌ی زیر را بیابید:
	\begin{displaymath}
		\frac{x^8 + 1}{(x^4 + 1)^2} = \frac{1}{2}
	\end{displaymath}
	
	\vspace{1cm}
	\hrulefill
	\vspace{1cm}
	
		% ------------------------------------------------------------------
	% سوال سی و دوم
	% ------------------------------------------------------------------
	\section*{سوال ۳۲: معادله گویا با ساختار متقارن}
	معادله‌ی زیر را برای تمام مقادیر حقیقی \(x\) حل کنید:
	\begin{displaymath}
		\frac{x^4 + 1}{x(x^2 + 1)} = \frac{41}{15}
	\end{displaymath}
	
	\vspace{1cm}
	\hrulefill
	\vspace{1cm}
	
		% ------------------------------------------------------------------
	% سوال سی و سوم (سوال بسیار چالشی‌تر)
	% ------------------------------------------------------------------
	\section*{سوال ۳۳: معادله رادیکالی با تقارن نامتقارن}
	تمام ریشه‌های حقیقی معادله‌ی زیر را بیابید:
	\begin{displaymath}
		\sqrt{x^2 - 3x + 1} + \sqrt{x} = \sqrt{x^2 + 1}
	\end{displaymath}
	
	\vspace{1cm}
	\hrulefill
	\vspace{1cm}
	
		% ------------------------------------------------------------------
	% سوال سی و چهارم
	% ------------------------------------------------------------------
	\section*{سوال ۳۴: دنباله حسابی از جنس جزء صحیح}
	تمام بازه‌های ممکن برای عدد حقیقی \(x\) را بیابید به طوری که اعداد \([x]\)، \([2x]\)، \([3x]\) و \([4x]\) چهار جمله‌ی متوالی یک دنباله حسابی باشند. (نماد \([y]\) نشان‌دهنده‌ی جزء صحیح عدد \(y\) است.)
	
	\vspace{1cm}
	\hrulefill
	\vspace{1cm}
	
		% ------------------------------------------------------------------
	% سوال سی و پنجم (سوال بسیار چالشی‌تر)
	% ------------------------------------------------------------------
	\section*{سوال ۳۵: دنباله‌های هم‌خط }
	یک دنباله‌ی حسابی \( (a_n) \) با قدر نسبت \(d\) و یک دنباله‌ی هندسی \( (b_n) \) با قدر نسبت \(r\) را در نظر بگیرید. برای هر عدد طبیعی \(n\)، سه نقطه در صفحه به صورت \(P_n = (a_n, b_n)\)، \(P_{n+1} = (a_{n+1}, b_{n+1})\) و \(P_{n+2} = (a_{n+2}, b_{n+2})\) تعریف می‌کنیم.
	\vspace{0.5cm}
	
	تمام شرایط ممکنی که بر روی این دو دنباله باید برقرار باشد را بیابید به طوری که به ازای \textbf{هر \(n \ge 1\)}، این سه نقطه همواره روی یک خط راست قرار بگیرند.
	
	\vspace{1cm}
	\hrulefill
	\vspace{1cm}
	
		% ------------------------------------------------------------------
	% سوال سی و ششم
	% ------------------------------------------------------------------
	\section*{سوال ۳۶: نردبانِ میانگین‌ها }
	پنج عدد حقیقی و متمایز \(a, b, c, d, e\) داده شده‌اند به طوری که سه شرط زیر به طور همزمان برقرار است:
	\begin{enumerate}[label=(\roman*)]
		\item اعداد \(a, b, c\) سه جمله‌ی متوالی یک \textbf{دنباله حسابی} هستند.
		\item اعداد \(c, d, e\) سه جمله‌ی متوالی یک \textbf{دنباله هندسی} هستند.
		\item اعداد \(a, c, e\) (جملات اول، وسط و آخر) خودشان سه جمله‌ی متوالی یک \textbf{دنباله هندسی} هستند.
	\end{enumerate}
	اگر بدانیم که \(a = 2\) و \(e = 18\) است، تمام مقادیر ممکن برای سه عدد \(b, c, d\) را بیابید.
	
	\vspace{1cm}
	\hrulefill
	\vspace{1cm}
	
		% ------------------------------------------------------------------
	% سوال سی و هفتم (سوال بسیار چالشی)
	% ------------------------------------------------------------------
	\section*{سوال ۳۷: دنباله‌های در هم تنیده }
	دو دنباله، یکی حسابی \( (a_n) \) با جمله‌ی اول \(a_1 = 2\) و قدر نسبت \(d\)، و دیگری هندسی \( (b_n) \) با جمله‌ی اول \(b_1 = 4\) و قدر نسبت \(r\) داده شده‌اند.
	\vspace{0.5cm}
	
	این دو دنباله به گونه‌ای "در هم تنیده" هستند که دو شرط زیر به طور همزمان برقرار است:
	\begin{enumerate}[label=(\roman*)]
		\item سه جمله‌ی \(a_1, b_1, a_2\) خودشان تشکیل یک \textbf{دنباله هندسی} می‌دهند.
		\item سه جمله‌ی \(b_1, a_2, b_2\) خودشان تشکیل یک \textbf{دنباله حسابی} می‌دهند.
	\end{enumerate}
	\vspace{0.5cm}
	
	مقدار جمله‌ی پنجم از دنباله‌ی جدید \( (c_n) \) که به صورت \(c_n = a_n + b_n\) تعریف می‌شود را بیابید.
	
	\vspace{1cm}
	\hrulefill
	\vspace{1cm}
	
	
		% ------------------------------------------------------------------
	% سوال سی و هشتم (سوال بسیار چالشی)
	% ------------------------------------------------------------------
	\section*{سوال ۳۸: دنباله ریشه‌های مثلثاتی}
	معادله درجه دو \( x^2 + px + q = 0 \) دارای دو ریشه‌ی حقیقی و متمایز \(\alpha\) و \(\beta\) است. این ریشه‌ها دو خاصیت منحصر به فرد زیر را دارند:
	\begin{enumerate}[label=(\roman*)]
		\item ریشه‌های \(\alpha\) و \(\beta\) برابر با سینوس و کسینوسِ یک زاویه‌ی یکسان \(\theta\) هستند. یعنی \( \{\alpha, \beta\} = \{\sin\theta, \cos\theta\} \).
		\item اعداد \(\alpha\)، \(\beta\) و \(\frac{1}{2}\) (به همین ترتیب) سه جمله‌ی متوالی یک \textbf{دنباله حسابی} را تشکیل می‌دهند.
	\end{enumerate}
	\vspace{0.5cm}
	
	تمام مقادیر ممکن برای ضریب \(p\) را بیابید.
	
	\vspace{1cm}
	\hrulefill
	\vspace{1cm}
	
		% ------------------------------------------------------------------
	% سوال سی و نهم
	% ------------------------------------------------------------------
	\section*{سوال ۳۹: تبدیل دنباله‌ها}
	سه عدد مثبت، سه جمله‌ی متوالی یک دنباله هندسی را تشکیل می‌دهند. اگر از جمله‌ی دوم ۴ واحد کم کنیم، سه عدد جدید تشکیل یک دنباله حسابی می‌دهند. حال اگر از جمله‌ی سومِ دنباله حسابیِ جدید، ۹ واحد دیگر کم کنیم، اعداد حاصل مجدداً تشکیل یک دنباله هندسی می‌دهند.
	\vspace{0.5cm}
	
	آن سه عدد اولیه را بیابید.
	
	\vspace{1cm}
	\hrulefill
	\vspace{1cm}
	
		% ------------------------------------------------------------------
	% سوال چهلم (سوال بسیار چالشی)
	% ------------------------------------------------------------------
	\section*{سوال ۴۰: همزیستی دنباله‌ها }
	مجموع سه جمله‌ی اول یک دنباله حسابی غیرثابت از اعداد حقیقی برابر با ۱۸ است و مجموع مربعات این سه جمله برابر با ۱۲۶ است.
	\vspace{0.5cm}
	
	سه عدد \(p, q, r\) را که خودشان تشکیل یک دنباله حسابی می‌دهند، به ترتیب به سه جمله‌ی دنباله‌ی حسابی اولیه اضافه می‌کنیم تا سه عدد جدید حاصل شود. این سه عدد جدید، یک \textbf{دنباله هندسی} را تشکیل می‌دهند.
	\vspace{0.5cm}
	
	اگر بدانیم که مجموع اعداد \(p, q, r\) برابر با صفر است (\(p+q+r=0\))، قدر نسبتِ دنباله هندسیِ نهایی را بیابید.
	
	\vspace{1cm}
	\hrulefill
	\vspace{1cm}
	
		% ------------------------------------------------------------------
	% سوال چهل و یکم
	% ------------------------------------------------------------------
	\section*{سوال ۴۱: دنباله حسابی-هندسی}
	چهار عدد حقیقی یک دنباله را تشکیل می‌دهند. سه جمله‌ی اول آن تشکیل یک دنباله حسابی و سه جمله‌ی آخر آن تشکیل یک دنباله هندسی می‌دهند. اگر مجموع دو جمله‌ی اول و آخر برابر با ۱۴ و مجموع دو جمله‌ی وسط برابر با ۱۲ باشد، آن چهار عدد را بیابید.
	
	\vspace{1cm}
	\hrulefill
	\vspace{1cm}
	
		% ------------------------------------------------------------------
	% سوال چهل و دوم (سوال بسیار چالشی)
	% ------------------------------------------------------------------
	\section*{سوال ۴۲: پلِ دنباله‌ها }
	چهار عدد حقیقی متمایز \(a, b, c, d\) دو شرط زیر را برآورده می‌کنند:
	\begin{enumerate}[label=(\roman*)]
		\item سه جمله‌ی \(a, b, c\) تشکیل یک \textbf{دنباله حسابی} می‌دهند.
		\item سه جمله‌ی \(b, c, d\) تشکیل یک \textbf{دنباله هندسی} می‌دهند.
	\end{enumerate}
	\vspace{0.5cm}
	
	علاوه بر این، یک پل ساختاری بین این دو دنباله برقرار است: قدر نسبت دنباله حسابی برابر با قدر نسبت دنباله هندسی است.
	\vspace{0.5cm}
	
	اگر مجموع این چهار عدد برابر با ۱۴ باشد (\(a+b+c+d=14\))، آن‌ها را بیابید.
	
	\vspace{1cm}
	\hrulefill
	\vspace{1cm}
	
		% ------------------------------------------------------------------
	% سوال چهل و سوم (سوال بسیار چالشی)
	% ------------------------------------------------------------------
	\section*{سوال ۴۳: دنباله با جملات مشترک}
	دو دنباله‌ی نامتناهی از اعداد طبیعی داده شده‌اند:
	\begin{enumerate}[label=(\roman*)]
		\item دنباله‌ی حسابی \( (A_m) \) که با \( A(m) = 4m - 1 \) برای \(m \ge 1\) تعریف می‌شود.
		\item دنباله‌ی درجه دوم \( (T_n) \) که با \( T(n) = n^2 - 2n + 4 \) برای \(n \ge 1\) تعریف می‌شود.
	\end{enumerate}
	\vspace{0.5cm}
	
	جملاتی که در هر دو دنباله \textbf{مشترک} هستند را در نظر بگیرید و آن‌ها را به ترتیب صعودی در یک دنباله‌ی جدید به نام \( (C_k) \) قرار دهید (\(k \ge 1\)).
	\vspace{0.5cm}
	
	\textbf{مطلوب است:}
	\begin{itemize}
		\item[الف)] ثابت کنید که دنباله‌ی جدید \( (C_k) \) خود یک دنباله‌ی درجه دوم است.
		\item[ب)] سه جمله‌ی اول دنباله‌ی \( (C_k) \) را بیابید.
		\item[ج)] جمله‌ی عمومی \( C(k) \) را بر حسب \(k\) پیدا کنید.
	\end{itemize}
	
	\vspace{1cm}
	
	\vspace{1cm}
	
		% ------------------------------------------------------------------
	% سوال چهل و چهارم (سوال بسیار چالشی)
	% ------------------------------------------------------------------
	\section*{سوال ۴۴: مارپیچ ناشناخته }
	یک ربات بر اساس یک دنباله‌ی درجه دوم ناشناخته به فرم \( T(n) = an^2 + bn + c \) حرکت می‌کند. ربات در مبدأ مختصات \( (0,0) \) شروع کرده و در \(n\) مرحله حرکت می‌کند.
	\begin{itemize}
		\item در مرحله ۱، به اندازه‌ی \(T(1)\) واحد به سمت \textbf{شرق} حرکت می‌کند.
		\item در مرحله ۲، به اندازه‌ی \(T(2)\) واحد به سمت \textbf{شمال} حرکت می‌کند.
		\item در مرحله ۳، به اندازه‌ی \(T(3)\) واحد به سمت \textbf{غرب} حرکت می‌کند.
		\item در مرحله ۴، به اندازه‌ی \(T(4)\) واحد به سمت \textbf{جنوب} حرکت می‌کند.
		\item و این الگوی چهار مرحله‌ای (شرق، شمال، غرب، جنوب) را ادامه می‌دهد.
	\end{itemize}
	\vspace{0.5cm}
	
	شما سه قطعه اطلاعات کلیدی در مورد این دنباله دارید:
	\begin{enumerate}[label=(\roman*)]
		\item طول اولین حرکت ربات برابر با ۳ واحد است (\(T(1) = 3\)).
		\item طول دومین حرکت ربات برابر با ۱۲ واحد است (\(T(2) = 12\)).
		\item پس از پایان مرحله سوم، مختصات ربات برابر با \( (-17, 12) \) است.
	\end{enumerate}
	\vspace{0.5cm}
	
	\textbf{مطلوب است:} مختصات نهایی ربات پس از پایان مرحله \textbf{ششم} چقدر است؟
	
	\vspace{1cm}
	\hrulefill
	\vspace{1cm}
	
		% ------------------------------------------------------------------
	% سوال چهل و پنجم (سوال بسیار چالشی)
	% ------------------------------------------------------------------
	\section*{سوال ۴۵: دنباله‌ی تفاضل‌های فروپاشیده}
	یک دنباله‌ی درجه دوم \( T(n) = an^2 + bn + c \) را در نظر بگیرید. دنباله‌ی تفاضل‌های آن را \(D(n)\) می‌نامیم، به طوری که \( D(n) = T(n+1) - T(n) \).
	\vspace{0.5cm}
	
	شما سه قطعه اطلاعات حیاتی در مورد این دنباله‌ها دارید:
	\begin{enumerate}[label=(\roman*)]
		\item سه جمله‌ی اول دنباله‌ی تفاضل‌ها، یعنی \(D(1), D(2), D(3)\)، خودشان تشکیل یک \textbf{دنباله هندسی} می‌دهند.
		\item اولین جمله‌ی دنباله‌ی تفاضل‌ها برابر با ۶ است (\(D(1) = 6\)).
		\item مجموع جملات اول و پنجم دنباله‌ی اصلی برابر با ۴۲ است (\(T(1) + T(5) = 42\)).
	\end{enumerate}
	\vspace{0.5cm}
	
	\textbf{مطلوب است:} \(S_{10}\)، یعنی مجموع ده جمله‌ی اول دنباله‌ی \(T(n)\) را بیابید.
	
	\vspace{1cm}
	\hrulefill
	\vspace{1cm}
	
		% ------------------------------------------------------------------
	% سوال چهل و ششم (سوال بسیار چالشی)
	% ------------------------------------------------------------------
	\section*{سوال ۴۶: نقاب نمایی }
	یک دنباله‌ی درجه دوم ناشناخته \( T(n) = an^2 + bn + c \) را در نظر بگیرید. دنباله‌ی جدیدی به نام \(G(n)\) را به صورت زیر تعریف می‌کنیم:
	\begin{displaymath}
		G(n) = 2^{T(n+1) - T(n)}
	\end{displaymath}
	\vspace{0.5cm}
	
	 همچنین دو اطلاعات زیر را در اختیار دارید:
	\begin{enumerate}[label=(\roman*)]
		\item اولین جمله‌ی دنباله‌ی \(G\) برابر با ۱۶ است (\(G(1) = 16\)).
		\item دومین جمله‌ی دنباله‌ی \(G\) برابر با ۲۵۶ است (\(G(2) = 256\)).
	\end{enumerate}
	\vspace{0.5cm}
	
	اگر بدانیم که اولین جمله‌ی دنباله‌ی اصلی برابر با ۵ است (\(T(1) = 5\))، مقدار جمله‌ی دهم آن، یعنی \(T(10)\)، را بیابید.
	
	\vspace{1cm}
	\hrulefill
	\vspace{1cm}
	
		% ------------------------------------------------------------------
	% سوال چهل و هفتم (با طعم جزء صحیح)
	% ------------------------------------------------------------------
	\section*{سوال ۴۷: دنباله‌ی گسسته }
	یک دنباله‌ی درجه دوم \( T(n) = \frac{n^2}{4} - \frac{n}{2} + 1 \) را در نظر بگیرید. دنباله‌ی جدیدی به نام \(B(n)\) را با استفاده از تابع جزء صحیح به صورت زیر تعریف می‌کنیم:
	\begin{displaymath}
		B(n) = [T(n)]
	\end{displaymath}
	(نماد \( [y] \) نشان‌دهنده‌ی جزء صحیح عدد \(y\) است.)
	\vspace{0.5cm}
	
	ثابت کنید دنباله‌ی \(B(n)\) \textbf{تقریباً} یک دنباله حسابی است و سپس مشخص کنید در کدام جملات، این دنباله از الگوی حسابی خود تخطی می‌کند. به عبارت دیگر، تمام مقادیر \(n\) را بیابید که به ازای آن‌ها، رابطه‌ی \( B(n+1) - B(n) \) با \( B(n) - B(n-1) \) برابر \textbf{نیست}.
	
	\vspace{1cm}
	\hrulefill
	\vspace{1cm}
	
		% ------------------------------------------------------------------
	% سوال چهل و هشتم (با طعم مثلثات)
	% ------------------------------------------------------------------
	\section*{سوال ۴۸: مارپیچ کسینوسی }
	یک دنباله‌ی حسابی \( \theta(n) \) از زاویه‌ها (بر حسب رادیان) با جمله‌ی اول \( \theta(1) = \frac{\pi}{3} \) و قدر نسبت \(d\) داده شده است. دنباله‌ی جدیدی به نام \(C(n)\) را به صورت زیر تعریف می‌کنیم:
	\begin{displaymath}
		C(n) = \cos(\theta(n))
	\end{displaymath}
	\vspace{0.5cm}
	
	شما می‌دانید که سه جمله‌ی اول دنباله‌ی \(C(n)\)، یعنی \(C(1), C(2), C(3)\)، خودشان تشکیل یک \textbf{دنباله هندسی} می‌دهند.
	\vspace{0.5cm}
	
	تمام مقادیر ممکن برای قدر نسبت \(d\) را در بازه‌ی \( (0, 2\pi] \) بیابید.
	
	\vspace{1cm}
	\hrulefill
	\vspace{1cm}
	
		% ------------------------------------------------------------------
	% سوال چهل و نهم (با طعم قدر مطلق)
	% ------------------------------------------------------------------
	\section*{سوال ۴۹: پیشروی بازتابی }
	سه عدد \(|x|\)، \(|x - 3|\) و \(|x - 6|\) (به همین ترتیب) سه جمله‌ی متوالی یک \textbf{دنباله هندسی} را تشکیل می‌دهند. تمام مقادیر حقیقی ممکن برای \(x\) را بیابید.
	
	\vspace{1cm}
	\hrulefill
	\vspace{1cm}
	
		% ------------------------------------------------------------------
	% سوال پنجاهم (با طعم مثلثات)
	% ------------------------------------------------------------------
	\section*{سوال ۵۰: آکوردِ پیشروی }
	سه زاویه‌ی متمایز \(A, B, C\) در یک \textbf{دنباله حسابی} قرار دارند. این زاویه‌ها به طور همزمان دو شرط زیر را برآورده می‌کنند:
	\begin{enumerate}[label=(\roman*)]
		\item دنباله‌ی \((\sin A, \sin B, \sin C)\) یک \textbf{دنباله هندسی} است.
		\item دنباله‌ی \((\cos A, \cos B, \cos C)\) یک \textbf{دنباله حسابی} است.
	\end{enumerate}
	\vspace{0.5cm}
	
	مقدار \( \cos(A-C) \) را بیابید.
	
	\vspace{1cm}
	\hrulefill
	\vspace{1cm}
	
		% ------------------------------------------------------------------
	% سوال پنجاه و یکم (با طعم جزء صحیح)
	% ------------------------------------------------------------------
	\section*{سوال ۵۱: راه‌پله‌ی خود-مرجع }
	یک دنباله‌ی حسابی از زوایا \( \theta(n) = a + (n-1)d \) و یک دنباله‌ی جدید \( B(n) = [10 \cdot \sin(\theta(n))] \) را در نظر بگیرید. می‌دانیم که:
	\begin{enumerate}[label=(\roman*)]
		\item دنباله‌ی \(B(n)\) خود یک دنباله حسابی با قدر نسبت \textbf{غیرصفر} \(D\) است.
		\item یک رابطه‌ی خود-مرجع بین این دو دنباله برقرار است: \( d = D \cdot \frac{\pi}{6} \).
		\item اولین جمله‌ی دنباله‌ی براکتی برابر با ۵ است، یعنی \(B(1) = 5\).
	\end{enumerate}
	\vspace{0.5cm}
	با فرض اینکه قدر نسبت \(D\) یک عدد صحیح است، مقدار \(B(3)\) را بیابید.
	
	\vspace{1cm}
	\hrulefill
	\vspace{1cm}
	
		% ------------------------------------------------------------------
	% سوال پنجاه و دوم (با طعم مثلثات)
	% ------------------------------------------------------------------
	\section*{سوال ۵۲: مجموع‌های گمشده در دایره}
	یک دنباله‌ی حسابی از زوایا \( \theta(n) = a + (n-1)d \) با جمله‌ی اول \(a = \frac{\pi}{12}\) و قدر نسبت \(d = \frac{\pi}{6}\) داده شده است.
	دو دنباله‌ی جدید \( C(n) = \cos(\theta(n)) \) و \( S(n) = \sin(\theta(n)) \) را تعریف می‌کنیم.
	همچنین، \( \text{Sum}(k) \) را برابر با مجموع \(k\) جمله‌ی اول دنباله‌ی \(C(n)\) در نظر می‌گیریم.
	\begin{displaymath}
		\text{Sum}(k) = \sum_{i=1}^{k} C(i)
	\end{displaymath}
	\vspace{0.5cm}
	
	اگر بدانیم که به ازای یک عدد طبیعی خاص \(k\)، مقدار \( \text{Sum}(k) \) برابر با \( \sin(\frac{k\pi}{12}) \) است، تمام مقادیر ممکن برای \(k\) را بیابید.
	
	\vspace{1cm}
	\hrulefill
	\vspace{1cm}
	
		% ------------------------------------------------------------------
	% سوال پنجاه و سوم (سوال بسیار چالشی)
	% ------------------------------------------------------------------
	\section*{سوال ۵۳: تقاطع دنباله‌ها}
	دو دنباله‌ی نامتناهی از اعداد طبیعی را در نظر بگیرید:
	\begin{enumerate}[label=(\roman*)]
		\item \textbf{دنباله‌ی درجه دوم \(T(n)\):} که با فرمول \( T(n) = n^2 + n + 1 \) تعریف می‌شود.
		\item \textbf{دنباله‌ی حسابی \(A(m)\):} که با فرمول \( A(m) = 4m + 3 \) تعریف می‌شود.
	\end{enumerate}
	\vspace{0.5cm}
	
	دنباله‌ی جدید \(C(k)\) را به عنوان دنباله‌ی جملات \textbf{مشترک} بین \(T(n)\) و \(A(m)\) تعریف می‌کنیم که به ترتیب صعودی مرتب شده‌اند.
	\vspace{0.5cm}
	
	\textbf{مطلوب است:} پنجاهمین جمله‌ی دنباله‌ی \(C(k)\)، یعنی \(C_{50}\)، را بیابید.
	
	\vspace{1cm}
	\hrulefill
	\vspace{1cm}
	
		% ------------------------------------------------------------------
	% سوال پنجاه و چهارم (نسخه‌ی کنکوری)
	% ------------------------------------------------------------------
	\section*{سوال ۵۴: پل بین دنباله و توان}
	در دنباله حسابی \( 7, 12, 17, \dots \)، چهارمین جمله‌ای که یک واحد از مربع کامل بیشتر است، کدام است؟
	
	\vspace{1cm}
	\hrulefill
	\vspace{1cm}
	
		% ------------------------------------------------------------------
	% سوال پنجاه و پنجم (سوال بسیار چالشی)
	% ------------------------------------------------------------------
	\section*{سوال ۵۵: معمای ضرایب هندسی}
	یک دنباله‌ی درجه سه به فرم \( T(n) = an^3 + bn^2 + cn + k \) تعریف شده است.
	ضرایب این دنباله، یعنی \(a, b, c, k\)، به همین ترتیب، چهار جمله‌ی متوالی یک \textbf{دنباله هندسی} هستند.
	\vspace{0.5cm}
	
	همچنین، دو قطعه اطلاعات کلیدی در مورد تفاضل‌های این دنباله در اختیار داریم:
	\begin{enumerate}[label=(\roman*)]
		\item تفاضل ثابت مرتبه‌ی سوم این دنباله برابر با ۴۸ است.
		\item اولین جمله‌ی دنباله‌ی تفاضل‌های مرتبه‌ی دوم، آن هم برابر با ۴۸ است.
	\end{enumerate}
	\vspace{0.5cm}
	
	اگر بدانیم که جمله‌ی اول این دنباله برابر با ۱۹ است (\(T(1) = 19\))، مقدار جمله‌ی دوم آن، یعنی \(T(2)\)، را بیابید.
	
	\vspace{1cm}
	\hrulefill
	\vspace{1cm}
	
		% ------------------------------------------------------------------
	% سوال پنجاه و ششم (بسیار بسیار چالشی)
	% ------------------------------------------------------------------
	\section*{سوال ۵۶: دنباله فراکتالی }
	یک دنباله‌ی درجه چهار به فرم \( T(n) = an^4 + bn^3 + cn^2 + dn + e \) را در نظر بگیرید.
	پنج ضریب این دنباله، یعنی \(a, b, c, d, e\)، به همین ترتیب، پنج جمله‌ی متوالی یک \textbf{دنباله حسابی} با قدر نسبت \(\delta\) هستند.
	\vspace{0.5cm}
	
	یک رابطه‌ی "فراکتالی"  بین این دو ساختار برقرار است:
	\begin{enumerate}[label=(\roman*)]
		\item تفاضل ثابت مرتبه‌ی چهارم دنباله‌ی \(T(n)\) برابر با ۷۲ است.
		\item قدر نسبت دنباله‌ی حسابیِ ضرایب (\(\delta\))، خود برابر با اولین جمله‌ی دنباله‌ی تفاضل‌های مرتبه دوم \(T(n)\) است. یعنی: \( \delta = \Delta^2T(1) \).
	\end{enumerate}
	\vspace{0.5cm}
	
	\textbf{مطلوب است:} مقدار اولین جمله‌ی دنباله‌ی تفاضل‌های مرتبه \textbf{سوم}، یعنی \( \Delta^3T(1) \)، را بیابید.
	\vspace{1cm}
	
	\paragraph{راهنمایی (فرمول‌های مورد نیاز):}
	برای یک دنباله‌ی درجه چهار، روابط زیر برقرار است:
	\begin{itemize}
		\item \( \Delta^4T(n) = 24a \)
		\item \( \Delta^2T(1) = 14a + 6b + 2c \)
		\item \( \Delta^3T(1) = 60a + 18b + 6c \)
	\end{itemize}
	
	\vspace{1cm}
	\hrulefill
	\vspace{1cm}
	
		% ------------------------------------------------------------------
	% سوال پنجاه و هفتم (با طعم نظریه اعداد)
	% ------------------------------------------------------------------
	\section*{سوال ۵۷: ریشه‌های صحیح مشترک}
	دو معادله درجه دو زیر را در نظر بگیرید:
	\begin{align*}
		x^2 + ax + b &= 0 \\
		x^2 + bx + a &= 0
	\end{align*}
	که در آن \(a\) و \(b\) اعداد صحیح و متمایز هستند.
	\vspace{0.5cm}
	
	اگر این دو معادله \textbf{دقیقاً یک ریشه مشترک} داشته باشند، مجموع تمام مقادیر ممکن برای \(a+b\) را بیابید.
	
	\vspace{1cm}
	\hrulefill
	\vspace{1cm}
	
		% ------------------------------------------------------------------
	% سوال پنجاه و هشتم (با طعم نامساوی‌ها)
	% ------------------------------------------------------------------
	\section*{سوال ۵۸: مینیممِ قدرمطلق}
	فرض کنید \(\alpha\) و \(\beta\) ریشه‌های حقیقی معادله‌ی \( x^2 - 2ax + a^2 - 4 = 0 \) باشند. کمترین مقدار ممکن برای عبارت \( |\alpha^3 - \beta^3| \) را بیابید.
	
	\vspace{1cm}
	\hrulefill
	\vspace{1cm}
	
		% ------------------------------------------------------------------
	% سوال پنجاه و نهم (با طعم هندسه تحلیلی)
	% ------------------------------------------------------------------
	\section*{سوال ۵۹: خط مماس مشترک }
	دو سهمی با معادلات \( y_1 = x^2 + ax + b \) و \( y_2 = cx^2 + dx + e \) را در نظر بگیرید.
	\vspace{0.5cm}
	
	این دو سهمی دارای یک \textbf{خط مماس مشترک} در نقطه‌ی \( (1, 1) \) هستند. علاوه بر این، می‌دانیم که رأس سهمی اول در نقطه‌ای با طول \(x=2\) قرار دارد و سهمی دوم از مبدأ مختصات \( (0,0) \) عبور می‌کند.
	\vspace{0.5cm}
	
	مقدار عبارت \( a + b + c + d + e \) را بیابید.
	
	\vspace{1cm}
	\hrulefill
	\vspace{1cm}
	
		% ------------------------------------------------------------------
	% سوال شصتم (بسیار بسیار چالشی)
	% ------------------------------------------------------------------
	\section*{سوال ۶۰: معادله‌ی خودآگاه}
	معادله‌ی درجه دو \( x^2 - px + q = 0 \) دارای دو ریشه‌ی حقیقی و متمایز است. ضرایب این معادله (\(p\) و \(q\)) و همچنین دلتای آن (\( \Delta = p^2 - 4q \))، سه ریشه‌ی معادله‌ی زیر هستند:
	\begin{displaymath}
		t^3 - 18t^2 + 101t - 180 = 0
	\end{displaymath}
	\vspace{0.5cm}
	
	تمام مقادیر ممکن برای ضریب \(p\) را بیابید.
	
	\vspace{1cm}
	\hrulefill
	\vspace{1cm}
	
		% ------------------------------------------------------------------
	% سوال شصت و یکم (بسیار بسیار چالشی)
	% ------------------------------------------------------------------
	\section*{سوال ۶۱: ریشه‌ی دلتای خودش}
	معادله‌ی درجه دو \( x^2 + px + q = 0 \) دارای دو خاصیت بسیار ویژه است:
	\begin{enumerate}[label=(\roman*)]
		\item جذرِ نامنفیِ دلتای آن (\( \sqrt{\Delta} \))، خود یکی از ریشه‌های همین معادله است.
		\item مجموع مربعات ریشه‌های آن برابر با ۳۶ است.
	\end{enumerate}
	\vspace{0.5cm}
	
	تمام مقادیر ممکن برای ضرایب \(p\) و \(q\) را بیابید.
	
	\vspace{1cm}
	\hrulefill
	\vspace{1cm}
	
		% ------------------------------------------------------------------
	% سوال شصت و دوم (بسیار بسیار چالشی)
	% ------------------------------------------------------------------
	\section*{سوال ۶۲: معادله با ضرایب بازگشتی}
	معادله‌ی درجه دو \( \Delta x^2 - (p+q)x + (p-q) = 0 \) را در نظر بگیرید. این معادله دارای سه خاصیت خارق‌العاده است:
	\begin{enumerate}[label=(\roman*)]
		\item پارامترهای \(p\) و \(q\) در این معادله، خودشان برابر با \textbf{ریشه‌های حقیقی و متمایز} همین معادله هستند.
		\item \( \Delta \) همان دلتای استاندارد معادله است.
		\item مقدار \(p\) از \(q\) بزرگتر است (\(p > q\)).
	\end{enumerate}
	\vspace{0.5cm}
	
	\textbf{مطلوب است:} مقادیر عددی \(p\) و \(q\) را بیابید.
	
	\vspace{1cm}
	\hrulefill
	\vspace{1cm}
	
		% ------------------------------------------------------------------
	% سوال شصت و سوم (بسیار بسیار چالشی)
	% ------------------------------------------------------------------
	\section*{سوال ۶۳: ریشه‌های رادیکالی}
	ریشه‌های حقیقی و متمایز \(\alpha\) و \(\beta\) از معادله‌ی درجه دو \( x^2 + px + q = 0 \)، در یک رابطه‌ی شگفت‌انگیز و متقارن با یکدیگر قرار دارند. آن‌ها در معادلات رادیکالی بی‌نهایت زیر صدق می‌کنند:
	\begin{align*}
		\alpha &= \sqrt{k + \beta} \\
		\beta &= \sqrt{k + \alpha}
	\end{align*}
	که در آن \(k\) یک ثابت حقیقی است.
	\vspace{0.5cm}
	
	اگر بدانیم که حاصلضرب ریشه‌های این معادله برابر با ۶- است (\(\alpha\beta = -6\))، مقادیر عددی \(p\)، \(q\) و \(k\) را بیابید.
	
	\vspace{1cm}
	\hrulefill
	\vspace{1cm}
	
		% ------------------------------------------------------------------
	% سوال شصت و چهارم (بسیار بسیار چالشی)
	% ------------------------------------------------------------------
	\section*{سوال ۶۴: دنباله‌ی مجموع توان‌ها}
	فرض کنید \(\alpha\) و \(\beta\) ریشه‌های معادله درجه دو \( x^2 - 3x - 5 = 0 \) باشند.
	دنباله‌ی \(S(n)\) را به صورت \( S(n) = \alpha^n + \beta^n \) برای \( n \ge 1 \) تعریف می‌کنیم.
	\vspace{0.5cm}
	
	\textbf{مطلوب است:} باقی‌مانده‌ی تقسیم \(S(10)\) بر عدد ۱۳ را بیابید.
	
	\vspace{1cm}
	\hrulefill
	\vspace{1cm}
	
		% ------------------------------------------------------------------
	% سوال شصت و ششم (با تغییر متغیر پنهان)
	% ------------------------------------------------------------------
	\section*{سوال 65: تقارن پنهان}
	معادله‌ی زیر را برای تمام مقادیر حقیقی \(x\) حل کنید:
	\begin{displaymath}
		(x-1)^4 + (x-3)^4 = 82
	\end{displaymath}
	
	\vspace{1cm}
	\hrulefill
	\vspace{1cm}
	
		% ------------------------------------------------------------------
	% سوال شصت و هفتم (با توان هفتم)
	% ------------------------------------------------------------------
	\section*{سوال 66: اتحاد گمشده‌ی توان هفتم}
	اعداد حقیقی \(x\) و \(y\) در دستگاه معادلات زیر صدق می‌کنند:
	\begin{displaymath}
		\begin{cases}
			x + y = 2 \\
			x^2 + y^2 = 6
		\end{cases}
	\end{displaymath}
	\vspace{0.5cm}
	
	مقدار عددی عبارت \( x^7 + y^7 \) را بیابید.
	
	\vspace{1cm}
	\hrulefill
	\vspace{1cm}
	
		% ------------------------------------------------------------------
	% سوال شصت و هشتم (با توان هفتم)
	% ------------------------------------------------------------------
	\section*{سوال 67: معادله‌ی تفاضل توان‌های هفتم}
	معادله‌ی زیر را برای تمام مقادیر حقیقی \(x\) حل کنید:
	\begin{displaymath}
		(x^2 - x + 1)^7 - (x^2 - x - 1)^7 = 128
	\end{displaymath}
	
	\vspace{1cm}
	\hrulefill
	\vspace{1cm}
	
		% ------------------------------------------------------------------
	% سوال شصت و نهم (با تغییر متغیر نامرئی)
	% ------------------------------------------------------------------
	\section*{سوال 68: معادله متقارن جابجا شده}
	معادله‌ی زیر را برای تمام مقادیر حقیقی \(x\) حل کنید:
	\begin{displaymath}
		(x^2 - 3x + 1)^2 - 5(x^2 - 3x + 1)(x+1) + 4(x+1)^2 = 0
	\end{displaymath}
	
	\vspace{1cm}
	\hrulefill
	\vspace{1cm}
	
		% ------------------------------------------------------------------
	% سوال هفتادم (بسیار بسیار چالشی)
	% ------------------------------------------------------------------
	
	% ------------------------------------------------------------------
	% سوال هفتادم (سوال درخواستی شما)
	% ------------------------------------------------------------------
	\section*{سوال 69: دنباله هندسی با جملات جبری}
	سه عبارت \( a^2+b \)، \( (a+b)^{\frac{3}{2}} \) و \( a+b^2 \) سه جمله‌ی متوالی یک دنباله هندسی هستند.
	به طوری که \(a\) و \(b\) اعداد صحیح غیرصفر می‌باشند (\(a, b \in \mathbb{Z} - \{0\}\)).
	چند مقدار ممکن برای حاصل‌ضرب \(ab\) وجود دارد؟
	
	\vspace{1cm}
	\hrulefill
	\vspace{1cm}
	
	% ------------------------------------------------------------------
	% سوال هفتاد و یکم (سوال چالشی پیشنهادی)
	% ------------------------------------------------------------------
	\section*{سوال 70: معمای هندسی دنباله‌ها}
	فرض کنید مجموعه \(S\) شامل تمام زوج‌مرتب‌های اعداد صحیح و \textbf{مثبت} \( (a,b) \) باشد که در آن، سه عبارت \( \log(a^2+b) \)، \( \log\left((a+b)^{3/2}\right) \) و \( \log(a+b^2) \) سه جمله‌ی متوالی یک \textbf{دنباله حسابی} باشند.
	
	\begin{enumerate}[label=(\alph*)]
		\item هر زوج‌مرتب \( (a,b) \in S \) را به عنوان یک نقطه در صفحه دکارتی در نظر بگیرید. ثابت کنید تمام این نقاط روی یک دایره قرار دارند (هم‌دایره هستند).
		\item معادله‌ی آن دایره را بیابید.
	\end{enumerate}
	
	\paragraph{راهنمایی:}
	به یاد بیاورید که اگر \( \log A, \log B, \log C \) یک دنباله حسابی تشکیل دهند، آنگاه اعداد \(A, B, C\) خودشان یک دنباله هندسی تشکیل می‌دهند.
	
	\vspace{1cm}
	\hrulefill
	\vspace{1cm}
	
	
	% ------------------------------------------------------------------
	% این خط را اینجا قرار دهید تا کد شما کامل باشد
	% ------------------------------------------------------------------
	% ------------------------------------------------------------------
	% سوال هفتاد و دوم (سوال درخواستی شما)
	% ------------------------------------------------------------------
	\section*{سوال 71: معادله‌ی رادیکالی-براکتی}
	معادله‌ی زیر را برای تمام مقادیر حقیقی \(x\) حل کنید:
	\begin{displaymath}
		[x] + \sqrt{x - \sqrt{x}} = \left[x + \frac{1}{x}\right]
	\end{displaymath}
	
	
	\vspace{1cm}
	\hrulefill
	\vspace{1cm}
	
	% ------------------------------------------------------------------
	% سوال هفتاد و سوم (سوال چالشی پیشنهادی)
	% ------------------------------------------------------------------
	\section*{سوال 72: معادله با تغییر متغیر پنهان}
	تمام ریشه‌های حقیقی معادله‌ی زیر را بیابید:
	\begin{displaymath}
		2\left(x + \frac{1}{x}\right) - 5\left(\sqrt{x} + \frac{1}{\sqrt{x}}\right) - 3 = 0
	\end{displaymath}
	
	
	
	\vspace{1cm}
	\hrulefill
	\vspace{1cm}
	
	
	% ------------------------------------------------------------------
	% این خط را اینجا قرار دهید تا کد شما کامل باشد
	% ------------------------------------------------------------------
	% ------------------------------------------------------------------
	% سوال هفتاد و دوم (سوال بسیار چالشی)
	% ------------------------------------------------------------------
	\section*{سوال 73: معادله‌ی خود-بازتابی}
	تمام ریشه‌های حقیقی \textbf{مثبت} معادله‌ی زیر را بیابید:
	\begin{displaymath}
		x[x] + \frac{1}{x} = x + \left[\frac{1}{x}\right]
	\end{displaymath}
	
	
	
	\vspace{1cm}
	\hrulefill
	\vspace{1cm}
	
	% ------------------------------------------------------------------
	% سوال هفتاد و سوم (سوال بسیار بسیار چالشی)
	% ------------------------------------------------------------------
	\section*{سوال 74: تحلیل پارامتری معادله‌ی براکتی}
	مجموعه‌ی تمام مقادیر حقیقی پارامتر \(a\) را بیابید که به ازای آن‌ها، معادله‌ی زیر \textbf{دقیقاً دو ریشه‌ی حقیقی متمایز} برای \(x\) داشته باشد:
	\begin{displaymath}
		[x]^2 - 3x + a = 0
	\end{displaymath}
	
	
	
	\vspace{1cm}
	\hrulefill
	\vspace{1cm}
	
	
	% ------------------------------------------------------------------
	% این خط را اینجا قرار دهید تا کد شما کامل باشد
	% ------------------------------------------------------------------

	% ------------------------------------------------------------------
	% سوال هفتاد و چهارم (سوال حل شده)
	% ------------------------------------------------------------------
	\section*{سوال75: دنیای زیبا(پیش نیاز سوال دانش اعداد مختلط میباشد)}
	تمام زوج‌مرتب‌های اعداد طبیعی \((\alpha, \beta)\) را بیابید به طوری که چندجمله‌ای \newline\(P(x) = (x + 2)^{\alpha} - (x^2 + 2)^{\beta}\) بر چندجمله‌ای \(Q(x) = x^2 - x + 3\) بخش‌پذیر باشد.
	
	
	
	\vspace{1cm}
	\hrulefill
	\vspace{1cm}
	
	% ------------------------------------------------------------------
	% سوال هفتاد و پنجم (سوال چالشی پیشنهادی)
	\section*{سوال 76 : دنیای زیبای بدون مختلط}
	تمام زوج‌مرتب‌های اعداد طبیعی \((\alpha, \beta)\) را بیابید به طوری که چندجمله‌ای \newline\(P(x) = (10 - 3x)^{\alpha} - (36 - 14x)^{\beta}\) بر چندجمله‌ای \(Q(x) = x^2 - 4\) بخش‌پذیر باشد.
	
	\vspace{1cm}
	\hrulefill
	\vspace{1cm}
	% ------------------------------------------------------------------
	\section*{سوال 77: معمای ریشه‌های مشترک}
	دو چندجمله‌ای \(P(x) = x^n - x + 2\) و \(Q(x) = x^2 - a\) را در نظر بگیرید که در آن \(n \ge 2\) یک عدد طبیعی و \(a\) یک پارامتر حقیقی است. اگر این دو چندجمله‌ای یک ریشه‌ی حقیقی \textbf{مشترک} داشته باشند، تمام مقادیر ممکن برای \(n\) را بیابید.
	
	
	\vspace{1cm}
	\hrulefill
	\vspace{1cm}
	
% ------------------------------------------------------------------
% سوال هفتاد و هفتم (بسیار چالشی)
% ------------------------------------------------------------------
\section*{سوال 78: معمای ریشه‌های صحیح}
به ازای چند مقدار صحیح برای پارامتر \(m\)، مجموعه‌ی تمام ریشه‌های معادله‌ی زیر، زیرمجموعه‌ای از مجموعه‌ی اعداد صحیح (\(\mathbb{Z}\)) است؟(تهی نیز زیرمجموعه اعداد صحیح است)
\begin{displaymath}
	x^4 - 9x^3 - 15x^2 + 4(m+36)x + 4m^2 = 0
\end{displaymath}


\vspace{1cm}
\hrulefill
\vspace{1cm}

% ------------------------------------------------------------------
% پاسخ تشریحی سوال هفتاد و هفتم
% ------------------------------------------------------------------

% ------------------------------------------------------------------
% سوال هفتاد و هشتم (بسیار بسیار چالشی)
% ------------------------------------------------------------------
\section*{سوال 79: معمای پارامتر و ریشه‌های صحیح}
تمام مقادیر صحیح پارامتر \(m\) را بیابید که به ازای آن‌ها، معادله‌ی زیر حداقل یک ریشه‌ی صحیح برای \(x\) داشته باشد.
\begin{displaymath}
	m^2 - 2x^2m + 10x^3 - 35x^2 + 50x - 24 = 0
\end{displaymath}

\vspace{1cm}
\hrulefill
\vspace{1cm}

% ------------------------------------------------------------------
% سوال هفتاد و نهم (بسیار چالشی با ایده هندسی)
% ------------------------------------------------------------------
% ------------------------------------------------------------------
% سوال هشتاد و یکم (بسیار بسیار چالشی)
% ------------------------------------------------------------------
\section*{سوال 80: معمای تقارن لگاریتمی}
دستگاه معادلات زیر را برای تمام زوج‌مرتب‌های حقیقی و مثبت \((x, y)\) حل کنید:
\begin{displaymath}
	\begin{cases}
		\log_y x - \log_x y = \frac{8}{3} \\
		\\
		xy = 16
	\end{cases}
\end{displaymath}


\vspace{1cm}
\hrulefill
\vspace{1cm}

% ------------------------------------------------------------------
% سوال هشتاد و دوم (بسیار بسیار چالشی)
% ------------------------------------------------------------------
\section*{سوال 81: معادله با هویت پنهان}
مجموعه‌ی تمام اعداد حقیقی \(x > 0\) که در معادله‌ی زیر صدق می‌کنند را بیابید.
\begin{displaymath}
	\{\log_2 x\} + \{\log_2 (1/x)\} = 1
\end{displaymath}
(نماد \( \{y\} \) نشان‌دهنده‌ی جزء کسری عدد \(y\) است که به صورت \( y - [y] \) تعریف می‌شود.)



\vspace{1cm}
\hrulefill
\vspace{1cm}

% ------------------------------------------------------------------
% سوال هشتاد و دوم (بسیار بسیار چالشی - تله مفهومی)
% ------------------------------------------------------------------
\section*{سوال ۸۲: معادله‌ی پوچ}
مجموعه‌ی جواب معادله‌ی زیر را در اعداد حقیقی بیابید.
\begin{displaymath}
	\Large \left[ \log_{\sin x} (\sqrt{2}) \right] + \left[ \log_{\cos x} (\sqrt{2}) \right] = 0
\end{displaymath}
(نماد \( [y] \) نشان‌دهنده‌ی جزء صحیح عدد \(y\) است.)

\vspace{1cm}
\hrulefill
\vspace{1cm}

% ------------------------------------------------------------------
% سوال هشتاد و سوم (شیطانی - نبرد توابع)
% ------------------------------------------------------------------
\section*{سوال ۸۳: نبرد توابع}
تمام ریشه‌های حقیقی معادله‌ی زیر را در بازه‌ی \( (0, 2\pi) \) بیابید.
\begin{displaymath}
	\Large x^{\left( [\log_2(\sin x)] + [\log_2(\cos x)] \right)} = \sin(2x)
\end{displaymath}

\vspace{1cm}

\vspace{1cm}
% ------------------------------------------------------------------
% سوال هشتاد و چهارم (بسیار بسیار چالشی - نبرد تیلور)
% ------------------------------------------------------------------
% ------------------------------------------------------------------
% سوال هشتاد و چهارم (بسیار بسیار چالشی - معمای پارامتر)
% ------------------------------------------------------------------
\section*{سوال ۸۴: معمای پارامتر}
مقدار ثابت و حقیقی \(a\) را طوری بیابید که حد زیر وجود داشته باشد و مقداری غیرصفر باشد. سپس مقدار آن حد را محاسبه کنید.
\begin{displaymath}
	\Large L = \lim_{x \to 0} \frac{a\sin(x) - \sin(ax)}{x^3}
\end{displaymath}


\vspace{1cm}
\hrulefill
\vspace{1cm}

% ------------------------------------------------------------------
% سوال هشتاد و پنجم (بسیار بسیار چالشی - کسینوس‌های زنجیره‌ای)
% ------------------------------------------------------------------
\section*{سوال ۸۵: کسینوس‌های زنجیره‌ای}
حاصل حد زیر را بیابید.
\begin{displaymath}
	\Large L = \lim_{x \to 0} \frac{1 - \cos(x)\cos(2x)\cos(3x)}{x^2}
\end{displaymath}

% ------------------------------------------------------------------
% سوال هشتاد و ششم (شیطانی - ماشینِ دقیق)
\vspace{1cm}
\hrulefill
\vspace{1cm}

% ------------------------------------------------------------------
\section*{سوال ۸۶: ماشینِ دقیق}
حاصل حد زیر را بیابید.
\begin{displaymath}
	\Large L = \lim_{x \to 0} \frac{\tan(x) + 2\sin(x) - 3x}{x^5}
\end{displaymath}



\vspace{1cm}
\hrulefill
\vspace{1cm}

% ------------------------------------------------------------------
% سوال هشتاد و هفتم (شیطانی - تفاوت بی‌نهایت کوچک‌ها)
% ------------------------------------------------------------------
\section*{سوال ۸۷: تفاوت بی‌نهایت کوچک‌ها}
حاصل حد زیر را محاسبه کنید.
\begin{displaymath}
	\Large L = \lim_{x \to 0} \frac{1 - \cos(1 - \cos(x)) - \frac{x^4}{8}}{x^6}
\end{displaymath}






\vspace{1cm}
\hrulefill
\vspace{1cm}


% ------------------------------------------------------------------
% سوال هشتاد و هشتم (بسیار چالشی - ساختار پنهان)
% ------------------------------------------------------------------
\section*{سوال ۸۸: ساختار پنهان}
فرض کنید در عبارت زیر، ضرایب \(a, b, c\) اعداد حقیقی غیرصفر باشند.
\begin{displaymath}
	L = \lim_{x \to b/a} \frac{ax^2+bx+c}{cx^2+bx+a}
\end{displaymath}
اگر بدانیم که این حد از نوع مبهم \( \frac{0}{0} \) است، حاصل حد، \(L\)، چند مقدار مختلف می‌تواند داشته باشد؟


\vspace{1cm}
\hrulefill
\vspace{1cm}

% ------------------------------------------------------------------
% سوال هشتاد و نهم (شیطانی - معمای حد متناهی)
% ------------------------------------------------------------------
% ------------------------------------------------------------------
% سوال هشتاد و نهم (بسیار چالشی - ساختار پنهان ۲)
% ------------------------------------------------------------------
\section*{سوال ۸۹: ساختار پنهان ۲}
فرض کنید \(a, b, c\) سه عدد حقیقی متمایز و ناصفر هستند.
\begin{displaymath}
	L = \lim_{x \to 1} \frac{(a-b)x^2 + (b-c)x + (c-a)}{x^2 - (a+1)x + a}
\end{displaymath}
اگر این حد موجود باشد، تمام مقادیر ممکن برای حاصل حد، \(L\)، را بیابید.



\vspace{1cm}
\hrulefill
\vspace{1cm}


\section*{سوال ۹۰: معمای دو نقطه‌ی مبهم}
فرض کنید چندجمله‌ای \(P(x) = x^3 + ax^2 + bx + c\) به گونه‌ای است که هر دو حد زیر موجود و متناهی هستند:
\begin{align*}
	L_1 &= \lim_{x \to 0} \frac{P(x)}{x} \\
	L_2 &= \lim_{x \to 1} \frac{P(x)}{x-1}
\end{align*}
اگر بدانیم که \(L_1 = 2L_2\)، مقدار \(b\) را بیابید.


\vspace{1cm}
\hrulefill
\vspace{1cm}
% ------------------------------------------------------------------
% سوال نود و یکم (بسیار چالشی - مزدوج پنهان)
% ------------------------------------------------------------------
\section*{سوال ۹۱: مزدوج پنهان}
حاصل حد زیر را بیابید.
\begin{displaymath}
	 L = \lim_{x \to 1} \frac{\sqrt{x} + \sqrt{x+3} - 3}{x^2 - 1}
\end{displaymath}



\vspace{1cm}
\hrulefill
\vspace{1cm}

% ------------------------------------------------------------------
% سوال نود و دوم (شیطانی - نبرد ریشه‌ها)
% ------------------------------------------------------------------
\section*{سوال ۹۲: نبرد ریشه‌ها}
حاصل حد زیر را محاسبه کنید.
\begin{displaymath}
 L = \lim_{x \to 1} \frac{\sqrt[3]{x} + \sqrt[3]{3x+5} - 3}{x^3 - 1}
\end{displaymath}


\vspace{1cm}
\hrulefill
\vspace{1cm}

% ------------------------------------------------------------------
% سوال نود و سوم (بسیار چالشی - تله پی)
% ------------------------------------------------------------------
\section*{سوال ۹۳: تله پی}
حاصل حد زیر را بیابید.
\begin{displaymath}
	\Large L = \lim_{x \to 0} \frac{\sin(\pi \cos^2(x))}{x^2}
\end{displaymath}



\vspace{1cm}
\hrulefill
\vspace{1cm}

% ------------------------------------------------------------------
% سوال نود و چهارم (بسیار بسیار چالشی - تله پی زنجیره‌ای)
% ------------------------------------------------------------------
\section*{سوال ۹۴: تله پی زنجیره‌ای}
حاصل حد زیر را محاسبه کنید.
\begin{displaymath}
	\Large L = \lim_{x \to 0} \frac{\sin(\pi \cos(x)\cos(2x))}{x^2}
\end{displaymath}



\vspace{1cm}
\hrulefill
\vspace{1cm}

% ------------------------------------------------------------------
% سوال نود و پنجم (شیطانی - آخرالزمان ریشه‌ها)
% ------------------------------------------------------------------
\section*{سوال ۹۵: آخرالزمان ریشه‌ها}
حاصل حد زیر را بیابید.
\begin{displaymath}
	 L = \lim_{x \to 0} \frac{\sqrt{\cos(x)} - \sqrt[3]{\cos(x)}}{\sin^2(x)}
\end{displaymath}



\vspace{1cm}
\hrulefill
\vspace{1cm}

% ------------------------------------------------------------------
% سوال نود و ششم (شیطانی - نقاب производная)
% ------------------------------------------------------------------
\section*{سوال ۹۶: نقاب(خارج از دبیرستان)}
حاصل حد زیر را محاسبه کنید که در آن \(a\) یک ثابت حقیقی در بازه‌ی \((0, \pi/2)\) است.
\begin{displaymath}
	\Large L = \lim_{x \to 0} \frac{\sin^x(a) + \cos^x(a) - 2}{x}
\end{displaymath}



\vspace{1cm}
\hrulefill
\vspace{1cm}

% ------------------------------------------------------------------
% سوال نود و هفتم (بسیار بسیار چالشی - فشار بی‌نهایت)
% ------------------------------------------------------------------
% ------------------------------------------------------------------
% سوال نود و هفتم (بسیار بسیار چالشی - فشار بی‌نهایت) (نسخه‌ی اصلاح‌شده)
% ------------------------------------------------------------------
\section*{سوال ۹۷: فشار بی‌نهایت}
حاصل حد زیر را بیابید.
\begin{displaymath}
	\Large L = \lim_{x \to \infty} \frac{[x]^2 + [2x]^2 + \dots + [nx]^2}{x^3}
\end{displaymath}
که در آن \(n\) یک عدد طبیعی ثابت است و \( [y] \) نشان‌دهنده‌ی جزء صحیح عدد \(y\) است.


\vspace{1cm}
\hrulefill
\vspace{1cm}

% ------------------------------------------------------------------
% سوال نود و هشتم (شیطانی - اره‌ی جزء صحیح)
% ------------------------------------------------------------------
\section*{سوال ۹۸: اره‌ی جزء صحیح}
حاصل حد زیر را محاسبه کنید.
\begin{displaymath}
	\Large L = \lim_{n \to \infty} \sum_{k=1}^{n} \frac{[kx]}{n^2}
\end{displaymath}
که در آن \(x\) یک عدد حقیقی ثابت است و \( [y] \) نشان‌دهنده‌ی جزء صحیح عدد \(y\) است.

% ------------------------------------------------------------------
% سوال نود و نهم (بسیار بسیار چالشی - حد دو چهره)
% ------------------------------------------------------------------
\section*{سوال ۹۹: حد دو چهره}
حاصل حد زیر را (در صورت وجود) بیابید.
\begin{displaymath}
	\Large L = \lim_{x \to 1} \frac{\sin(x^2 - \lfloor x \rfloor)}{x - \lfloor x^2 \rfloor}
\end{displaymath}
(نماد \( \lfloor y \rfloor \) نشان‌دهنده‌ی جزء صحیح عدد \(y\) است.)

\vspace{1cm}
\hrulefill
\vspace{1cm}

% ------------------------------------------------------------------
% سوال صدم (شیطانی - پژواک جزء کسری)
% ------------------------------------------------------------------
\section*{سوال ۱۰۰: پژواک جزء کسری}
حاصل حد زیر را (در صورت وجود) محاسبه کنید.
\begin{displaymath}
	\Large L = \lim_{x \to 1} \frac{\sin(x - \lfloor x \rfloor)}{\tan(x^2 - \lfloor x^2 \rfloor)}
\end{displaymath}

\vspace{1cm}
\hrulefill
\vspace{1cm}


% ------------------------------------------------------------------
% سوال صد و یکم (بسیار چالشی - کشف پارامتر)
% ------------------------------------------------------------------
\section*{سوال ۱۰۱: کشف پارامتر}
دو پارامتر حقیقی \(a\) و \(b\) به گونه‌ای هستند که حد زیر برقرار است. مقدار \(ab\) را بیابید.
\begin{displaymath}
	\Large \lim_{x \to b} \frac{1}{ax^2 - 4x + a + 3} = -\infty
\end{displaymath}

\vspace{1cm}
\hrulefill
\vspace{1cm}

% ------------------------------------------------------------------
% سوال صد و دوم (بسیار بسیار چالشی - مسیر دوگانه)
% ------------------------------------------------------------------
\section*{سوال ۱۰۲: مسیر دوگانه}
سه پارامتر حقیقی \(a, b, c\) به گونه‌ای هستند که حد زیر برقرار است.
\begin{displaymath}
	\Large \lim_{x \to b} \frac{x-c}{ax^2 - 4x + a + 3} = +\infty
\end{displaymath}
اگر بدانیم که \(b < 0\) و \(c\) یک عدد صحیح است، کمترین مقدار ممکن برای \(c\) چقدر است؟

\vspace{1cm}
\hrulefill
\vspace{1cm}

% ------------------------------------------------------------------
% سوال صد و سوم (شیطانی - ریشه‌های تکراری)
% ------------------------------------------------------------------
\section*{سوال ۱۰۳: ریشه‌های تکراری}
به ازای چند مقدار حقیقی برای پارامتر \(a\)، یک عدد حقیقی \(b\) وجود دارد به طوری که حد زیر برقرار باشد؟
\begin{displaymath}
	\Large \lim_{x \to b} \frac{x^2+1}{3x^4 - 4x^3 - 12x^2 + a} = -\infty
\end{displaymath}

\vspace{1cm}
\hrulefill
\vspace{1cm}

% ------------------------------------------------------------------
% سوال صد و چهارم (آخرالزمانی - معمای توان)
% ------------------------------------------------------------------
\section*{سوال ۱۰۴: معمای توان}
برای هر یک از توان‌های \(n \in \{3, 4, 6, 8\}\)، بررسی کنید که آیا مقدار حقیقی و منحصر به فردی برای پارامتر \(a\) وجود دارد که به ازای آن، حد زیر برای یک \(b\) حقیقی برقرار باشد.
\begin{displaymath}
	\Large \lim_{x \to b} \frac{-1}{a - (x^2-4)^n} = +\infty
\end{displaymath}
در نهایت، مجموع تمام مقادیر \(a\) که برای این توان‌ها پیدا می‌کنید را محاسبه کنید.

\vspace{1cm}
\hrulefill
\vspace{1cm}

% ------------------------------------------------------------------
% سوال صد و پنجم (بسیار چالشی - نبرد سینوس و کسینوس)
% ------------------------------------------------------------------
\section*{سوال ۱۰۵: نبرد سینوس و کسینوس}
حاصل حد زیر را بیابید.
\begin{displaymath}
	\Large \lim_{x \to 0} \frac{\sin(x) - x\cos(x)}{x^3}
\end{displaymath}

\vspace{1cm}
\hrulefill
\vspace{1cm}

% ------------------------------------------------------------------
% سوال صد و ششم (شیطانی - هویت پنهان کمان دو برابر)
% ------------------------------------------------------------------
\section*{سوال ۱۰۶: هویت پنهان کمان دو برابر}
حاصل حد زیر را محاسبه کنید.
\begin{displaymath}
	\Large \lim_{x \to 0} \frac{2\sin(x) - \sin(2x)}{x^3}
\end{displaymath}

% ------------------------------------------------------------------
% سوال صد و هفتم (بسیار چالشی - اتحاد گمشده)
% ------------------------------------------------------------------
\section*{سوال ۱۰۷: اتحاد گمشده}
حاصل حد زیر را بیابید.
\begin{displaymath}
	\Large \lim_{x \to 0} \frac{\tan(5x) - \tan(3x) - \tan(2x)}{x^3}
\end{displaymath}

\vspace{1cm}
\hrulefill
\vspace{1cm}

% ------------------------------------------------------------------
% سوال صد و هشتم (شیطانی - اتحاد تودرتو)
% ------------------------------------------------------------------
\section*{سوال ۱۰۸: اتحاد تودرتو}
حاصل حد زیر را محاسبه کنید.
\begin{displaymath}
	\Large \lim_{x \to 0} \frac{\tan(5x) - \tan(3x) - \tan(2x) - 30x^3}{x^5}
\end{displaymath}

\vspace{1cm}
\hrulefill
\vspace{1cm}
% ------------------------------------------------------------------
% سوال صد و نهم (بسیار چالشی - معمای مجانب)
% ------------------------------------------------------------------
\section*{سوال ۱۰۹: مشتق پنهان}
فرض کنید \(g(x)\) تابع وارونِ تابع \( f(x) = x^5 + 2x^3 + x - 4 \) باشد. حاصل حد زیر را بیابید.
\begin{displaymath}
	\Large L = \lim_{h \to 0} \frac{g(h) - g(0)}{h}
\end{displaymath}

\vspace{1cm}
\hrulefill
\vspace{1cm}

% ------------------------------------------------------------------
% سوال صد و دهم (جدید - شیطانی)
% ------------------------------------------------------------------
\section*{سوال ۱۱۰: نبرد وارون و جزء صحیح}
فرض کنید \(g(x)\) تابع وارونِ تابع \( f(x) = x^3 + x - 10 \) باشد. حاصل حد زیر را (در صورت وجود) محاسبه کنید.
\begin{displaymath}
	\Large L = \lim_{x \to 0} \frac{\lfloor g(x) \rfloor \cdot (g(x) - g(0))}{x}
\end{displaymath}
(نماد \( \lfloor y \rfloor \) نشان‌دهنده‌ی جزء صحیح عدد \(y\) است.)

\vspace{1cm}
\hrulefill
\vspace{1cm}
% ------------------------------------------------------------------
% سوال صد و یازدهم (سوال درخواستی شما)
% ------------------------------------------------------------------
\section*{سوال ۱۱۱: شرط پیوستگی سراسری}
محدوده‌ی تمام مقادیر حقیقی پارامتر \(a\) را بیابید که به ازای آن‌ها، تابع زیر بر روی تمام اعداد حقیقی (\(\mathbb{R}\)) پیوسته باشد.
\begin{displaymath}
	f(x) = \frac{\sin(x)}{2 + a\cos(x)}
\end{displaymath}

\vspace{1cm}
\hrulefill
\vspace{1cm}

% ------------------------------------------------------------------
% سوال صد و دوازدهم (سوال چالشی پیشنهادی)
% ------------------------------------------------------------------
\section*{سوال ۱۱۲: معمای پیوستگی و مشتق‌پذیری}
دو پارامتر حقیقی \(a\) و \(b\) را طوری بیابید که تابع دوتکه‌ای زیر، در نقطه‌ی \(x = \frac{\pi}{4}\) \textbf{مشتق‌پذیر} باشد.
\begin{displaymath}
	f(x) = 
	\begin{cases}
		\tan(x) &  x < \frac{\pi}{4} \\
		a\sin(x) + b\cos(x) &  x \ge \frac{\pi}{4}
	\end{cases}
\end{displaymath}



\vspace{1cm}
\hrulefill
\vspace{1cm}
% ------------------------------------------------------------------
% سوال صد و سیزدهم (نسخه‌ی چالشی سوال درخواستی)
% ------------------------------------------------------------------
\section*{سوال ۱۱۳: معمای ریشه‌های پنهان}
تابع \( f(x) = (x^2 + ax + b) [x^3] \) را در نظر بگیرید. می‌دانیم که این تابع در بازه‌ی \( (0, \sqrt[3]{4}) \) \textbf{دقیقاً یک نقطه ناپیوستگی} دارد. بیشترین مقدار ممکن برای پارامتر \(b\) کدام است؟
\begin{itemize}[rightmargin=2cm]
	\item نماد \( [y] \) نشان‌دهنده‌ی جزء صحیح عدد \(y\) است.
\end{itemize}



\vspace{1cm}
\hrulefill
\vspace{1cm}

% ------------------------------------------------------------------
% سوال صد و چهاردهم (سوال بسیار چالشی پیشنهادی)
% ------------------------------------------------------------------
\section*{سوال ۱۱۴: شمارش ناپیوستگی‌های خنثی‌شده}
تعداد نقاط ناپیوستگی تابع زیر را در بازه‌ی باز \( (0, 5) \) بیابید.
\begin{displaymath}
	f(x) = (x^2 - 7x + 12) \left[ \frac{x^2}{2} - x \right]
\end{displaymath}



\vspace{1cm}
\hrulefill
\vspace{1cm}

% ------------------------------------------------------------------
% سوال صد و پانزدهم (سوال درخواستی شما - با رویکرد مثلثاتی)
% ------------------------------------------------------------------
\section*{سوال ۱۱۵: حد بنیادی }
حاصل حد زیر که در آن \(a\) یک ثابت حقیقی است، همواره کدام است؟
\begin{displaymath}
	L = \lim_{x \to a} \frac{\cos(x) - \cos(a)}{x - a}
\end{displaymath}



\vspace{1cm}
\hrulefill
\vspace{1cm}

% ------------------------------------------------------------------
% سوال صد و شانزدهم (بسیار چالشی - با ایده زنجیره‌ای)
% ------------------------------------------------------------------
\section*{سوال ۱۱۶: حد زنجیره‌ای}
حاصل حد زیر را بر حسب ثابت حقیقی \(a\) بیابید.
\begin{displaymath}
	L = \lim_{x \to a} \frac{\cos(\sin(x)) - \cos(\sin(a))}{x - a}
\end{displaymath}



\vspace{1cm}
\hrulefill
\vspace{1cm}


% ------------------------------------------------------------------
% سوال صد و هفدهم (نسخه‌ی چالشی سوال درخواستی)
% ------------------------------------------------------------------
\section*{سوال ۱۱۷: معمای حد در مرزهای گسسته}
تابع \(f(x)\) به صورت زیر تعریف شده است. اگر این تابع در تمام نقاط دامنه‌اش \textbf{دارای حد} باشد، مقدار \( \frac{b}{a} \) را بیابید.
\begin{displaymath}
	f(x) = 
	\begin{cases}
		ax^2 + bx - 3 &  1 < [x] \le 3 \\
		4ax + 3b      &  [x] \le 1 \\
		4bx + a       &  [x] > 3
	\end{cases}
\end{displaymath}


\vspace{1cm}
\hrulefill
\vspace{1cm}

% ------------------------------------------------------------------
% سوال صد و هجدهم (بسیار چالشی - اتصال نرم)
% ------------------------------------------------------------------
\section*{سوال ۱۱۸: اتصال نرم}
چهار پارامتر حقیقی \(a, b, c, d\) را به گونه‌ای بیابید که تابع سه‌ضابطه‌ای زیر در تمام دامنه‌اش \textbf{مشتق‌پذیر} باشد.
\begin{displaymath}
	f(x) = 
	\begin{cases}
		x^3 + 3x^2 + 2x & x < 0 \\
		ax^2 + bx + c   &  0 \le x < 1 \\
		dx - 2          &  x \ge 1
	\end{cases}
\end{displaymath}
در نهایت، مقدار \(a+b+c+d\) را محاسبه کنید.



\vspace{1cm}
\hrulefill
\vspace{1cm}

% ------------------------------------------------------------------
% سوال صد و نوزدهم (نسخه‌ی چالشی سوال درخواستی)
% ------------------------------------------------------------------
\section*{سوال ۱۱۹: معمای رادیکال‌های تو در تو}
حاصل حد زیر را بیابید.
\begin{displaymath}
	L = \lim_{x \to 2\pi} \frac{\sqrt{\cos(x)}\sqrt[3]{\cos(2x)} - 1}{\sin(x)\sin(2x)}
\end{displaymath}



\vspace{1cm}
\hrulefill
\vspace{1cm}

% ------------------------------------------------------------------
% سوال صد و بیستم (بسیار چالشی - نبرد اتحادها)
% ------------------------------------------------------------------
\section*{سوال ۱۲۰: نبرد اتحادها}
حاصل حد زیر را محاسبه کنید.
\begin{displaymath}
	L = \lim_{x \to 0} \frac{1 - \cos(x)\cos(2x) - \tan^2(x)}{x \sin(2x)}
\end{displaymath}



\vspace{1cm}
\hrulefill
\vspace{1cm}

% ------------------------------------------------------------------
% سوال صد و بیست و یکم (سوال درخواستی شما)
% ------------------------------------------------------------------
\section*{سوال ۱۲۱: معمای حد در نقاط صحیح}
به ازای عدد صحیح \(k\)، تابع \( f(x)=2x[x] - k^2[-x] \) در \(x=k\) حد دارد. مجموع مقادیر قابل قبول برای \(k\) کدام است؟



\vspace{1cm}
\hrulefill
\vspace{1cm}

% ------------------------------------------------------------------
% سوال صد و بیست و دوم (بسیار چالشی - هارمونی اعداد)
% ------------------------------------------------------------------
\section*{سوال ۱۲۲: هارمونی اعداد}
دو ثابت حقیقی \(a\) و \(b\) به گونه‌ای انتخاب شده‌اند که تابع زیر، در \textbf{تمام نقاط صحیح}، پیوسته باشد. زوج مرتب \( (a,b) \) را بیابید.
\begin{displaymath}
	f(x) = [x^2] + a[x]^2 + b[x]
\end{displaymath}



\vspace{1cm}
\hrulefill
\vspace{1cm}

% ------------------------------------------------------------------
% سوال صد و بیست و سوم (شیطانی - حدی که وجود نداشت)
% ------------------------------------------------------------------
\section*{سوال ۱۲۳: حدی که وجود نداشت}
مجموعه‌ی تمام مقادیر صحیح پارامتر \(k\) را بیابید که به ازای آن‌ها، تابع زیر در نقطه‌ی \(x=2\) دارای حد باشد.
\begin{displaymath}
	f(x) = \frac{[x]^2 - k[x]}{[x] - 2}
\end{displaymath}



\vspace{1cm}
\hrulefill
\vspace{1cm}

% ------------------------------------------------------------------
% سوال صد و بیست و سوم (شیطانی - جایگزین سوال قبلی)
% ------------------------------------------------------------------
\section*{سوال 124: توهم وجود}
تمام مقادیر \textbf{صحیح} پارامتر \(k\) را بیابید که به ازای آن‌ها، حد تابع زیر در نقطه‌ی \(x=k\) موجود و متناهی باشد.
\begin{displaymath}
	f(x) = \frac{ [x]^3 - (k+2)[x]^2 + 2k[x] }{ [x] - k }
\end{displaymath}

\vspace{1cm}
\hrulefill
\vspace{1cm}

% ------------------------------------------------------------------
% سوال صد و بیست و چهارم (بسیار بسیار چالشی - دروازه‌های دوگانه)
% ------------------------------------------------------------------
\section*{سوال 125: دروازه‌های دوگانه}
پارامترهای حقیقی \(a\) و \(b\) را طوری بیابید که تابع زیر، هم در نقطه‌ی \(x=2\) و هم در نقطه‌ی \(x=-2\) دارای حد باشد.
\begin{displaymath}
	f(x) = \frac{[x]^3 + a[x]^2 + b[x]}{[x]^2 - 4}
\end{displaymath}

\vspace{1cm}
\hrulefill
\vspace{1cm}
% ------------------------------------------------------------------
% سوال صد و بیست و پنجم (نسخه‌ی چالشی سوال درخواستی)
% ------------------------------------------------------------------
\section*{سوال 126: نبرد ریشه‌ها و توان‌ها}
اگر حد زیر موجود و برابر با \( \frac{3}{2} \) باشد، مقدار عدد طبیعی \(n\) کدام است؟
\begin{displaymath}
	L = \lim_{x \to 1} \frac{\sqrt[3]{x^n} - 1}{\sqrt{x^2 - 1} - \sqrt{x-1}}
\end{displaymath}

\vspace{1cm}
\hrulefill
\vspace{1cm}

% ------------------------------------------------------------------
% سوال صد و بیست و ششم (بسیار چالشی - معمای تانژانت)
% ------------------------------------------------------------------
\section*{سوال 127: معمای تانژانت}
مقدار ثابت و مثبت \(n\) به گونه‌ای است که حاصل حد زیر برابر با ۳۲ است. مقدار \(n\) را بیابید.
\begin{displaymath}
	L = \lim_{x \to 0} \frac{\tan(nx) - \sin(nx)}{x^3}
\end{displaymath}

\vspace{1cm}
\hrulefill
\vspace{1cm}

% ------------------------------------------------------------------
% سوال صد و بیست و هفتم (شیطانی - دستگاه حدود)
% ------------------------------------------------------------------
\section*{سوال 128: دستگاه حدود}
دو پارامتر طبیعی و متمایز \(n\) و \(m\) در دو حد زیر صدق می‌کنند. این دو پارامتر را بیابید.
\begin{align*}
	L_1 &= \lim_{x \to 1} \frac{\sqrt{x^n} - 1}{x^m - 1} = 2 \\
	L_2 &= \lim_{x \to 1} \frac{x^n - 1}{\sqrt[3]{x^m} - 1} = 12
\end{align*}

\vspace{1cm}
\hrulefill
\vspace{1cm}


% ------------------------------------------------------------------
% سوال صد و بیست و هشتم (نسخه‌ی چالشی سوال درخواستی)
% ------------------------------------------------------------------
\section*{سوال ۱۲۸: شمارش معکوس ناپیوستگی‌ها}
تعداد نقاط ناپیوستگی تابع زیر را در بازه‌ی \( (\frac{1}{9}, 9) \) بیابید.
\begin{displaymath}
	f(x) = (\log_2 x) \cdot [\log_2 x]
\end{displaymath}

\vspace{1cm}
\hrulefill
\vspace{1cm}

% ------------------------------------------------------------------
% سوال صد و بیست و نهم (بسیار چالشی - معمای ضرایب گمشده)
% ------------------------------------------------------------------
\section*{سوال ۱۲۹: معمای ضرایب گمشده}
تابع \( f(x) = (x^2 - ax + b)[\log_2 x] \) را در نظر بگیرید. این تابع در بازه‌ی \( (1, 10) \) دقیقاً یک نقطه ناپیوستگی دارد. مجموع تمام مقادیر ممکن برای پارامتر \(a\) را بیابید.

\vspace{1cm}
\hrulefill
\vspace{1cm}

% ------------------------------------------------------------------
% سوال صد و سی‌ام (نسخه‌ی چالشی سوال درخواستی)
% ------------------------------------------------------------------
\section*{سوال ۱۳۰: موازنه‌ی توان‌ها}
اگر حد زیر موجود و حاصل آن برابر با \(2\sqrt{2}\) باشد، مقدار \(m-n\) را بیابید. (\(m\) و \(n\) اعداد حقیقی هستند)
\begin{displaymath}
	L = \lim_{x \to \frac{\pi}{4}} \frac{(1-\tan x)^n}{(\cos x - \sin x)^m}
\end{displaymath}

\vspace{1cm}
\hrulefill
\vspace{1cm}

% ------------------------------------------------------------------
% سوال صد و سی و یکم (شیطانی - مجموع زنجیره‌ای)
% ------------------------------------------------------------------
\section*{سوال ۱۳۱: مجموع زنجیره‌ای}
اگر حاصل حد زیر برابر با \( \frac{55}{2} \) باشد، مقدار عدد طبیعی \(n\) کدام است؟
\begin{displaymath}
	L = \lim_{x \to 0} \frac{1 - \cos(x)\cos(2x)\cos(3x)\cdots\cos(nx)}{x^2}
\end{displaymath}

\vspace{1cm}
\hrulefill
\vspace{1cm}

% ------------------------------------------------------------------
% سوال صد و سی و دوم (سوال درخواستی ۱)
% ------------------------------------------------------------------
\section*{سوال ۱۳۲: پیوستگی در مرز بی‌نهایت}
اگر تابع دوضابطه‌ای زیر در نقطه‌ی \(x=\frac{\pi}{2}\) پیوسته باشد، حاصل \( \frac{a}{b} \) کدام است؟
\begin{displaymath}
	f(x)=
	\begin{cases}
		\frac{\sqrt{2a-4x}}{\sqrt{\cos x}-\sqrt{\cos(3x)}} & ; x < \frac{\pi}{2} \\
		b-\sin x & ; x \ge \frac{\pi}{2}
	\end{cases}
\end{displaymath}

\vspace{1cm}
\hrulefill
\vspace{1cm}

% ------------------------------------------------------------------
% سوال صد و سی و سوم (سوال درخواستی ۲)
% ------------------------------------------------------------------
\section*{سوال ۱۳۳: حد رادیکال‌های متقاطع}
حاصل حد زیر کدام است؟
\begin{displaymath}
	L = \lim_{x \to \frac{\pi}{4}^-} \sqrt{\frac{1-\tan^2 x}{\sqrt{1-\sin(2x)}}}
\end{displaymath}

\vspace{1cm}
\hrulefill
\vspace{1cm}

% ------------------------------------------------------------------
% سوال صد و سی و چهارم (سوال درخواستی ۳)
% ------------------------------------------------------------------
\section*{سوال ۱۳۴: معمای پیوستگی حاصل‌ضرب}
دو تابع \(f(x)\) و \(g(x)\) به صورت زیر تعریف شده‌اند:
\begin{align*}
	f(x) &= \begin{cases} x^2 &; x < 1 \\ ax+b &; x \ge 1 \end{cases} \\
	g(x) &= \begin{cases} ax+2 &; x < 2 \\ x+4 &; x \ge 2 \end{cases}
\end{align*}
اگر تابع حاصل‌ضرب \( (f \cdot g)(x) \) روی تمام اعداد حقیقی پیوسته باشد، بیشترین مقدار ممکن برای \(a+b\) کدام است؟

\vspace{1cm}
\hrulefill
\vspace{1cm}

% ------------------------------------------------------------------
% سوال صد و سی و پنجم (نسخه‌ی چالشی سوال ۱۳۲)
% ------------------------------------------------------------------
\section*{سوال ۱۳۵: پیوستگی در مرز بی‌نهایت ۲}
دو پارامتر حقیقی و ناصفر \(a\) و \(b\) به گونه‌ای هستند که تابع زیر در \(x=\frac{\pi}{2}\) پیوسته است. مقدار \(ab\) را بیابید.
\begin{displaymath}
	f(x)=
	\begin{cases}
		\frac{\sqrt[3]{a - 2\sin x}}{\cos x} & ; x \ne \frac{\pi}{2} \\
		b & ; x = \frac{\pi}{2}
	\end{cases}
\end{displaymath}

\vspace{1cm}
\hrulefill
\vspace{1cm}

% ------------------------------------------------------------------
% سوال صد و سی و ششم (نسخه‌ی چالشی سوال ۱۳۳)
% ------------------------------------------------------------------
\section*{سوال ۱۳۶: حد رادیکال‌های شیطانی}
حاصل حد زیر را بیابید.
\begin{displaymath}
	L = \lim_{x \to 0} \frac{\sqrt{1+x\sin x} - \sqrt{\cos(2x)}}{x^2}
\end{displaymath}

\vspace{1cm}
\hrulefill
\vspace{1cm}

% ------------------------------------------------------------------
% سوال صد و سی و هفتم (بسیار بسیار چالشی - معمای پیوستگی ترکیب توابع)
% ------------------------------------------------------------------
\section*{سوال ۱۳۷: پیوستگی در قلب ماشین}
دو تابع \(f(x)\) و \(g(x)\) به صورت زیر تعریف شده‌اند:
\begin{align*}
	f(x) &= \begin{cases} x-1 &; x \ge 2 \\ 2x-3 &; x < 2 \end{cases} \\
	g(x) &= \begin{cases} ax+b &; x \ge 1 \\ x^2 &; x < 1 \end{cases}
\end{align*}
اگر تابع مرکب \( (f \circ g)(x) \) روی تمام اعداد حقیقی پیوسته باشد، تمام زوج‌مرتب‌های ممکن برای \( (a,b) \) را بیابید.

\vspace{1cm}
\hrulefill
\vspace{1cm}

% ------------------------------------------------------------------
% سوال صد و سی و هشتم (شیطانی - پیوستگی وارون)
% ------------------------------------------------------------------
\section*{سوال ۱۳۸: پیوستگی وارون}
تابع دوتکه‌ای زیر را در نظر بگیرید:
\begin{displaymath}
	f(x) = 
	\begin{cases}
		x^2 - 4x + 6 & ; x \ge 2 \\
		-x+4 & ; x < 2
	\end{cases}
\end{displaymath}
مجموعه‌ی تمام مقادیر حقیقی \(a\) را بیابید که به ازای آن‌ها، تابع \( g(x) = [f(x)] + [f^{-1}(x)] - a \) در نقطه‌ی \(x=3\) پیوسته باشد.

\vspace{1cm}
\hrulefill
\vspace{1cm}

% ------------------------------------------------------------------
% سوال صد و سی و نهم (بسیار چالشی - حاصل‌ضرب همزیست)
% ------------------------------------------------------------------
\section*{سوال ۱۳۹: حاصل‌ضرب همزیست}
تابع \(f(x)\) به صورت زیر تعریف شده است:
\begin{displaymath}
	f(x) = 
	\begin{cases}
		(x-a)^2 & ; x \ge a \\
		-x+1 & ; x < 1
	\end{cases}
\end{displaymath}
اگر تابع \(f(x)\) در دامنه‌اش یک به یک باشد و تابع جدید \(h(x) = f(x) \cdot f^{-1}(x)\) در نقطه‌ی \(x=1\) پیوسته باشد، مقدار \(a\) را بیابید.

\vspace{1cm}
\hrulefill
\vspace{1cm}

% ------------------------------------------------------------------
% سوال صد و چهلم (شیطانی - عدد گمشده)
% ------------------------------------------------------------------
\section*{سوال ۱۴۰: عدد گمشده}
تابع \(h(x)\) برای تمام اعداد حقیقی نامنفی (\(x \ge 0\)) به صورت زیر تعریف می‌شود:
\begin{displaymath}
	h(x) = [x^3] + [\sqrt[3]{x}]
\end{displaymath}
کوچکترین عدد صحیح \textbf{مثبت} \(k\) را بیابید که در برد تابع \(h(x)\) قرار \textbf{نداشته باشد}.
(به عبارت دیگر، کوچکترین \( k \in \mathbb{Z}^+ \) که به ازای هیچ \(x \ge 0\)، تساوی \(h(x)=k\) برقرار نباشد.)

\vspace{1cm}
\hrulefill
\vspace{1cm}

% ------------------------------------------------------------------
% سوال صد و چهل و یکم (آخرالزمانی - جهان‌های موازی)
% ------------------------------------------------------------------
\section*{سوال ۱۴۱: جهان‌های موازی}
تابع خطی \(f(x) = ax+b\) یک تابع وارون‌پذیر است. نمودارهای \(y=f(x)\) و \(y=f^{-1}(x)\) دو خط راست \textbf{متمایز و موازی} هستند. اگر فاصله‌ی بین این دو خط برابر با \(2\sqrt{2}\) باشد، تمام مقادیر ممکن برای پارامتر \(b\) را بیابید.

\vspace{1cm}
\hrulefill
\vspace{1cm}

% ------------------------------------------------------------------
% سوال صد و چهل و دوم (نسخه‌ی چالشی سوال درخواستی)
% ------------------------------------------------------------------
\section*{سوال ۱۴۲: معمای حد چندوجهی}
اگر حد زیر موجود و حاصل آن یک عدد حقیقی ناصفر \(a\) باشد، مقدار \(a^n\) کدام است؟
\begin{displaymath}
	L = \lim_{x \to 0^+} \frac{\sin(\sqrt{1-x^3}-1) - 2\tan([x])}{x^n(1-\cos(\sqrt{3x}))}
\end{displaymath}

\vspace{1cm}
\hrulefill
\vspace{1cm}

% ------------------------------------------------------------------
% سوال صد و چهل و سوم (بسیار چالشی - تله‌ی بی‌نهایت کوچک)
% ------------------------------------------------------------------
\section*{سوال ۱۴۳: تله‌ی بی‌نهایت کوچک}
مقدار پارامترهای حقیقی \(n\) و \(a \ne 0\) را طوری بیابید که حد زیر موجود و برابر با \(a\) باشد. سپس مقدار \( \frac{n}{a} \) را محاسبه کنید.
\begin{displaymath}
	L = \lim_{x \to 0^+} \frac{\sqrt{1+\sin(x^2)} - \cos(x)}{x^n}
\end{displaymath}

\vspace{1cm}
\hrulefill
\vspace{1cm}

% ------------------------------------------------------------------
% سوال صد و چهل و چهارم (شیطانی - موازنه‌ی جزء کسری)
% ------------------------------------------------------------------
\section*{سوال ۱۴۴: موازنه‌ی جزء کسری}
اگر حد زیر موجود و حاصل آن یک عدد حقیقی ناصفر \(a\) باشد، مقدار \(an\) را بیابید.
\begin{displaymath}
	L = \lim_{x \to 0^+} \frac{\{x\} - \tan(\{x\})}{x^n}
\end{displaymath}
(نماد \( \{y\} \) نشان‌دهنده‌ی جزء کسری عدد \(y\) است که به صورت \( y - [y] \) تعریف می‌شود.)

\vspace{1cm}
\hrulefill
\vspace{1cm}

% ------------------------------------------------------------------
% سوال صد و چهل و پنجم (آخرالزمانی - نبرد دو جبهه)
% ------------------------------------------------------------------
% ------------------------------------------------------------------
% سوال صد و چهل و پنجم (آخرالزمانی - نبرد دو جبهه - نسخه‌ی اصلاح‌شده)
% ------------------------------------------------------------------
\section*{سوال ۱۴۵: نبرد دو جبهه}
حد زیر را در نظر بگیرید که در آن \(m, n, p, q\) اعداد حقیقی ناصفر هستند.
\begin{displaymath}
	L = \lim_{x \to 0} \frac{\cos(mx) - \cos(nx)}{\cos(px) - \cos(qx)}
\end{displaymath}
اگر بدانیم که \(n^2-m^2 = 8\) و حاصل حد، \(L\)، برابر با \(2\) است، مقدار عبارت \(p^2-q^2\) را بیابید.

\vspace{1cm}
\hrulefill
\vspace{1cm}

% ------------------------------------------------------------------
% سوال صد و چهل و ششم (جایگزین سوال ۱۴۱ - حد دنباله‌ای از توابع)
% ------------------------------------------------------------------
\section*{سوال ۱۴۶: حد شبح‌وار}
تابع \(f(x)\) به عنوان حد نقطه‌ای زیر تعریف می‌شود. تعداد نقاط ناپیوستگی تابع \(f(x)\) را در بازه‌ی بسته \( [0, 2\pi] \) بیابید.
\begin{displaymath}
	f(x) = \lim_{n \to \infty} (\sin^2 x)^n
\end{displaymath}
(توجه: در اینجا \(n\) به سمت بی‌نهایت میل می‌کند، نه \(x\)).

\vspace{1cm}
\hrulefill
\vspace{1cm}

% ------------------------------------------------------------------
% سوال صد و چهل و هفتم (شیطانی - پیوستگی در نقطه‌ی گنگ)
% ------------------------------------------------------------------
\section*{سوال ۱۴۷: پیوستگی در نقطه‌ی گنگ}
تابع زیر را در نظر بگیرید:
\begin{displaymath}
	f(x) = 
	\begin{cases}
		x^2 & ; x \in \mathbb{Q} \\
		2x-1 & ; x \notin \mathbb{Q}
	\end{cases}
\end{displaymath}
در چند نقطه بر روی محور اعداد حقیقی، این تابع پیوسته است؟

\vspace{1cm}
\hrulefill
\vspace{1cm}

% ------------------------------------------------------------------
% سوال صد و چهل و هشتم (آخرالزمانی - معمای حد ضربی)
% ------------------------------------------------------------------
\section*{سوال ۱۴۸: معمای حد ضربی}
حد زیر را محاسبه کنید.
\begin{displaymath}
	L = \lim_{n \to \infty} \prod_{k=1}^{n} \cos\left(\frac{x}{2^k}\right)
\end{displaymath}
(نماد \( \prod \) نشان‌دهنده‌ی حاصل‌ضرب جملات است.)

\vspace{1cm}
\hrulefill
\vspace{1cm}

% ------------------------------------------------------------------
% سوال صد و چهل و نهم (بسیار چالشی - حد کسینوسی)
% ------------------------------------------------------------------
\section*{سوال ۱۴۹: رقص کسینوسی در بی‌نهایت}
تابع \(f(x)\) به صورت حد زیر تعریف شده است. مجموع طول بازه‌هایی که تابع \(f(x)\) روی آن‌ها تعریف شده و پیوسته است را در بازه‌ی اصلی \( [0, 2\pi] \) بیابید.
\begin{displaymath}
	f(x) = \lim_{n \to \infty} (\cos x)^n
\end{displaymath}

\vspace{1cm}
\hrulefill
\vspace{1cm}

% ------------------------------------------------------------------
% سوال صد و پنجاهم (شیطانی - پل بینهایت)
% ------------------------------------------------------------------
\section*{سوال ۱۵۰: پل بی‌نهایت}
تابع \(f(x)\) به صورت حد زیر تعریف شده است. این تابع در بازه‌ی \( (0, \infty) \) در چند نقطه ناپیوسته است؟
\begin{displaymath}
	f(x) = \lim_{n \to \infty} \frac{x^{2n} - \cos(\frac{\pi}{x})}{x^{2n} + \cos(\frac{\pi}{x})}
\end{displaymath}

\vspace{1cm}
\hrulefill
\vspace{1cm}

% ------------------------------------------------------------------
% سوال صد و پنجاه و یکم (آخرالزمانی - جزء صحیح در افق)
% ------------------------------------------------------------------
% ------------------------------------------------------------------
% سوال صد و پنجاه و یکم (آخرالزمانی - جزء صحیح در افق - نسخه‌ی اصلاح‌شده)
% ------------------------------------------------------------------
\section*{سوال ۱۵۱: جزء صحیح در افق}
برای هر عدد حقیقی \(x\)، تابع \(f(x)\) به صورت حد نقطه‌ای زیر تعریف می‌شود. مقدار عبارت \newline\( f(\sqrt{2}) + f(\pi) \) را بیابید.
\begin{displaymath}
	f(x) = \lim_{n \to \infty} \frac{[nx]}{n}
\end{displaymath}
(نماد \( [y] \) نشان‌دهنده‌ی جزء صحیح عدد \(y\) است.)

\vspace{1cm}
\hrulefill
\vspace{1cm}

% ------------------------------------------------------------------
% سوال صد و پنجاه و دوم (معمای حد و علامت - نسخه‌ی نهایی)
% ------------------------------------------------------------------

\section*{سوال ۱۵۲: معمای حد و علامت}
حد زیر را در نظر بگیرید که در آن \(a \ne 0\).
\begin{displaymath}
	L = \lim_{x \to a} \frac{x^3 - (2a-1)x^2 + (a^2-2a)x + a^2}{x^2 - a^2}
\end{displaymath}
اگر حاصل حد، \(L\)، یک عدد حقیقی باشد و در نامساوی \( a \cdot L \le 0 \) صدق کند، بزرگترین مقدار ممکن برای \(a\) را بیابید.

\vspace{1cm}
\hrulefill
\vspace{1cm}

% ------------------------------------------------------------------
% سوال صد و پنجاه و سوم (جایگزین - بسیار چالشی)
% ------------------------------------------------------------------
\section*{سوال ۱۵۳: دامنه مثلثاتی}
مجموعه‌ی تمام مقادیر حقیقی پارامتر \(m\) را بیابید که به ازای آن‌ها، دامنه‌ی تابع زیر، تمام اعداد حقیقی (\(\mathbb{R}\)) باشد.
\begin{displaymath}
	f(x) = \frac{1}{\sqrt{\sin^2(x) + m\sin(x) + 2}}
\end{displaymath}

\vspace{1cm}
\hrulefill
\vspace{1cm}

% ------------------------------------------------------------------
% سوال صد و پنجاه و چهارم (پیوستگی در آشیانه - نسخه‌ی نهایی)
% ------------------------------------------------------------------
\section*{سوال ۱۵۴: پیوستگی در آشیانه}
تابع \( f(x) = \left[\frac{1}{2}[x^2-2x]\right] \) در بازه‌ی \( (a, 1] \) پیوسته است. بیشترین مقدار ممکن برای \(a\) کدام است؟

\vspace{1cm}
\hrulefill
\vspace{1cm}

% ------------------------------------------------------------------
% سوال صد و پنجاه و پنجم (دستگاه حدود - نسخه‌ی نهایی)
% ------------------------------------------------------------------
\section*{سوال ۱۵۵: دستگاه حدود}
دو حد زیر را در نظر بگیرید که در آن \( a \notin \{0, 1\} \).
\begin{align*}
	L_1 &= \lim_{x \to a} \frac{x^2 - (a+1)x + a}{x^2 - a^2} \\
	L_2 &= \lim_{x \to 1} \frac{x^2 - (a+1)x + a}{x-1}
\end{align*}
اگر هر دو حد موجود باشند و رابطه‌ی \(2L_1 = L_2\) برقرار باشد، مقدار پارامتر \(a\) را بیابید.

\vspace{1cm}
\hrulefill
\vspace{1cm}

% ------------------------------------------------------------------
% سوال صد و پنجاه و ششم (مخرج خاموش - نسخه‌ی نهایی)
% ------------------------------------------------------------------
\section*{سوال ۱۵۶: مخرج خاموش}
به ازای چند مقدار حقیقی برای پارامتر \(a\)، تابع زیر در دامنه‌اش دقیقاً دو نقطه‌ی بحرانی (ریشه‌های مشتق) دارد؟
\begin{displaymath}
	f(x) = \frac{1}{x^4 - 8x^2 + a}
\end{displaymath}

\vspace{1cm}
\hrulefill
\vspace{1cm}

% ------------------------------------------------------------------
% سوال صد و پنجاه و هفتم (مارپیچ براکتی - نسخه‌ی نهایی)
% ------------------------------------------------------------------
\section*{سوال ۱۵۷: مارپیچ براکتی}
تابع \( g(x) = x^2 - 4x + 3 \) را در نظر بگیرید. تعداد نقاط ناپیوستگی تابع مرکب \( f(x) = [g(x)] \cdot \{g(x)\} \) در بازه‌ی \( [0, 4] \) را بیابید.
(نمادهای \( [y] \) و \( \{y\} \) به ترتیب نشان‌دهنده‌ی جزء صحیح و جزء کسری عدد \(y\) هستند.)

\vspace{1cm}
\hrulefill
\vspace{1cm}

% ------------------------------------------------------------------
% سوال صد و پنجاه و هشتم (نسخه‌ی چالشی سوال درخواستی)
% ------------------------------------------------------------------
\section*{سوال ۱۵۸: هارمونی ناپیوستگی}
تابع \( f(x) = \{x\} - \{-x\} \) را در نظر بگیرید. حد چپ این تابع در هر یک از نقاط ناپیوستگی‌اش روی خط \(y=m\) و حد راست آن در همین نقاط روی خط \(y=n\) قرار می‌گیرد. مقدار \(m+n\) کدام است؟
(نماد \( \{y\} \) نشان‌دهنده‌ی جزء کسری عدد \(y\) است.)

\vspace{1cm}
\hrulefill
\vspace{1cm}

% ------------------------------------------------------------------
% سوال صد و پنجاه و نهم (شیطانی - پیوستگی خودخفته)
% ------------------------------------------------------------------
\section*{سوال ۱۵۹: پیوستگی خودخفته}
تابع \( f(x) = [2x] - 2[x] \) را در نظر بگیرید. تعداد نقاط ناپیوستگی تابع مرکب \( g(x) = f(f(x)) \) در بازه‌ی \( [0, 10] \) کدام است؟

\vspace{1cm}
\hrulefill
\vspace{1cm}

% ------------------------------------------------------------------
% سوال صد و شصتم (آخرالزمانی - سایه‌ی سهمی)
% ------------------------------------------------------------------
\section*{سوال ۱۶۰: سایه‌ی سهمی}
تابع \( h(x) = [x^2] - [x]^2 \) در تمام نقاط صحیح دارای پیوستگی راست است. مقدار حد چپ این تابع در نقطه‌ی \(x=k\) (که \(k\) یک عدد صحیح و \(k>1\) است) برابر با \(m_k\) می‌باشد. اگر بدانیم \(m_k = 20\)، مقدار \(k\) کدام است؟

\vspace{1cm}
\hrulefill
\vspace{1cm}

% ------------------------------------------------------------------
% سوال صد و شصت و یکم (سوال درخواستی شما)
% ------------------------------------------------------------------
\section*{سوال ۱۶۱: مسابقه‌ی توابع}
تعداد نقاط ناپیوستگی تابع \( y=\max\{[x], \sin x\} \) در بازه‌ی بسته \( [-3, 3] \) کدام است؟

\vspace{1cm}
\hrulefill
\vspace{1cm}

% ------------------------------------------------------------------
% سوال صد و شصت و دوم (بسیار چالشی - تفاضل قدرمطلقی)
% ------------------------------------------------------------------
\section*{سوال ۱۶۲: تفاضل قدرمطلقی}
تابع \( h(x) = \max\{[x], \{x\}\} - \min\{[x], \{x\}\} \) را در نظر بگیرید. مجموع تمام مقادیر صحیح \(k\) در بازه‌ی \( [-10, 10] \) که این تابع در آن‌ها پیوسته است، کدام است؟
(نمادهای \( [y] \) و \( \{y\} \) به ترتیب نشان‌دهنده‌ی جزء صحیح و جزء کسری عدد \(y\) هستند.)

\vspace{1cm}
\hrulefill
\vspace{1cm}

% ------------------------------------------------------------------
% سوال صد و شصت و سوم (شیطانی - دوئل ماکسیمم‌ها)
% ------------------------------------------------------------------
\section*{سوال ۱۶۳: دوئل ماکسیمم‌ها}
مجموعه‌ی تمام مقادیر متمایزی که تابع زیر می‌تواند اختیار کند (برد تابع) را بیابید.
\begin{displaymath}
	f(x) = [\max(x, -x)] - \max([x], [-x])
\end{displaymath}

\vspace{1cm}
\hrulefill
\vspace{1cm}

% ------------------------------------------------------------------
% سوال صد و شصت و چهارم (آخرالزمانی - پیوستگی پارامتری)
% ------------------------------------------------------------------
\section*{سوال ۱۶۴: پیوستگی پارامتری}
مجموعه‌ی تمام مقادیر حقیقی پارامتر \(a\) را بیابید که به ازای آن‌ها، تابع زیر در نقطه‌ی \(x=1\) پیوسته باشد.
\begin{displaymath}
	f(x) = \min\{[x+a], x^2\}
\end{displaymath}

\vspace{1cm}
\hrulefill
\vspace{1cm}

% ------------------------------------------------------------------
% سوال صد و شصت و پنجم (سوال درخواستی شما - نسخه‌ی دقیق)
% ------------------------------------------------------------------
\section*{سوال ۱۶۵: معمای پیوستگی زوج و فرد}
تابع \(f(x)\) به صورت زیر تعریف شده است.
\begin{displaymath}
	f(x) = 
	\begin{cases}
		|x - [x]| & ;  [x] \text{ زوج باشد} \\
		|x - [x-a]| & ;  [x] \text{ فرد باشد}
	\end{cases}
\end{displaymath}
اگر این تابع روی تمام اعداد حقیقی پیوسته باشد و بدانیم که \(a\) یک عدد ثابت است، تمام مقادیر ممکن برای \( [a] \) را بیابید.

\vspace{1cm}
\hrulefill
\vspace{1cm}

% ------------------------------------------------------------------
% سوال صد و شصت و ششم (بسیار چالشی - تابع خودآگاه)
% ------------------------------------------------------------------
\section*{سوال ۱۶۶: تابع خودآگاه}
تابع \(f(x)\) به صورت زیر تعریف شده است.
\begin{displaymath}
	f(x) = 
	\begin{cases}
		ax+b & ;  [f(x)] \text{ زوج باشد} \\
		-x+4 & ;  [f(x)] \text{ فرد باشد}
	\end{cases}
\end{displaymath}
اگر این تابع روی تمام اعداد حقیقی پیوسته باشد، مقدار \( a-b \) کدام است؟

\vspace{1cm}
\hrulefill
\vspace{1cm}

% ------------------------------------------------------------------
% سوال صد و شصت و هفتم (آخرالزمانی - مارپیچ بازگشتی)
% ------------------------------------------------------------------
\section*{سوال ۱۶۷: مارپیچ بازگشتی}
تابع \(f(x)\) به صورت زیر تعریف شده است.
\begin{displaymath}
	f(x) = 
	\begin{cases}
		x/2 & ;  [x] \text{ زوج باشد} \\
		x-1 & ; [x] \text{ فرد باشد}
	\end{cases}
\end{displaymath}
کوچکترین عدد صحیح \(k>1\) که در معادله‌ی \( f(f(k)) = 1 \) صدق می‌کند، کدام است؟

\vspace{1cm}
\hrulefill
\vspace{1cm}

% ------------------------------------------------------------------
% سوال صد و شصت و هشتم (بسیار چالشی - معمای سه‌گانه)
% ------------------------------------------------------------------
\section*{سوال ۱۶۸: معمای سه‌گانه}
تابع \(f(x)\) به صورت زیر تعریف شده است.
\begin{displaymath}
	f(x) = 
	\begin{cases}
		[x] + a & ; [x]\cdot f(x) < 0 \\
		x+b & ;  [x]\cdot f(x) = 0 \\
		[x]+c & ;  [x]\cdot f(x) > 0
	\end{cases}
\end{displaymath}
اگر این تابع در تمام نقاط صحیح \textbf{مثبت}، پیوسته باشد، مقدار \(a+b+c\) را بیابید.

\vspace{1cm}
\hrulefill
\vspace{1cm}

% ------------------------------------------------------------------
% سوال صد و شصت و نهم (آخرالزمانی - هزارتوی پیوستگی)
% ------------------------------------------------------------------
\section*{سوال ۱۶۹: هزارتوی پیوستگی}
تابع \(f(x)\) به صورت زیر تعریف شده است.
\begin{displaymath}
	f(x) = 
	\begin{cases}
		ax-1 & ;  [x] \text{ اول باشد} \\
		bx+c & ;  [x] \text{ مرکب باشد} \\
		x^2 & ; \text{در سایر حالات}
	\end{cases}
\end{displaymath}
اگر این تابع در تمام نقاط صحیح \( k \ge 2 \) پیوسته باشد، مقدار \(a+b+c\) را بیابید.

\vspace{1cm}
\hrulefill
\vspace{1cm}

% ------------------------------------------------------------------
% سوال صد و هفتادم (آخرالزمانی - پیوستگی متقاطع)
% ------------------------------------------------------------------

% ------------------------------------------------------------------
% سوال صد و هفتاد و یکم (آخرالزمانی - پیوستگی دوگانه)
% ------------------------------------------------------------------
\section*{سوال 170: پیوستگی دوگانه}
دو تابع \(f(x) = [x]^2 + ax\) و \(g(x) = [x^2] + b\) را در نظر بگیرید. تابع \(h(x)\) به صورت زیر تعریف شده است:
\begin{displaymath}
	h(x) = 
	\begin{cases}
		f(x) & ; [x] \text{ زوج باشد} \\
		g(x) & ;  [x] \text{ فرد باشد}
	\end{cases}
\end{displaymath}
اگر تابع \(h(x)\) هم در نقطه‌ی \(x=2\) و هم در نقطه‌ی \(x=3\) پیوسته باشد، مقدار \(a-b\) را بیابید.

\vspace{1cm}
\hrulefill
\vspace{1cm}
% ------------------------------------------------------------------
% سوال صد و هفتاد و دوم (معمای پیوستگی - نسخه‌ی نهایی)
% ------------------------------------------------------------------
\section*{سوال ۱۷۲: معمای پیوستگی زوج و فرد}
فرض کنید \(n\) یک عدد طبیعی ثابت باشد و تابع \(f(x)\) به صورت زیر تعریف شود:
\begin{displaymath}
	f(x) = 
	\begin{cases}
		x + n & ;  [x] \text{ زوج باشد} \\
		-x + n[x] & ;  [x] \text{ فرد باشد}
	\end{cases}
\end{displaymath}
اگر این تابع در نقطه‌ی صحیح \(x=k\) پیوسته باشد، تمام مقادیر ممکن برای \(k\) را بیابید.

\vspace{1cm}
\hrulefill
\vspace{1cm}

% ------------------------------------------------------------------
% سوال صد و هفتاد و سوم (تابع آونگ - نسخه‌ی نهایی)
% ------------------------------------------------------------------
\section*{سوال ۱۷۳: تابع آونگ}
تابع \(f(x)\) به صورت زیر تعریف شده است.
\begin{displaymath}
	f(x) = 
	\begin{cases}
		ax - [x] & ;  [x] \text{ زوج باشد} \\
		bx + [x]^2 & ; [x] \text{ فرد باشد}
	\end{cases}
\end{displaymath}
اگر این تابع هم در نقطه‌ی \(x=1\) و هم در نقطه‌ی \(x=2\) پیوسته باشد، مقدار \(a+b\) کدام است؟

\vspace{1cm}
\hrulefill
\vspace{1cm}

% ------------------------------------------------------------------
% سوال صد و هفتاد و چهارم (ناپیوستگی پارامتری - نسخه‌ی نهایی)
% ------------------------------------------------------------------
\section*{سوال ۱۷۴: ناپیوستگی پارامتری}
فرض کنید \(n\) یک پارامتر حقیقی است.
\begin{displaymath}
	f(x) = 
	\begin{cases}
		\max\{n, x\} & ;  [x] \text{ زوج باشد} \\
		\min\{n, x\} & ;  [x] \text{ فرد باشد}
	\end{cases}
\end{displaymath}
مجموعه‌ی تمام مقادیر \(n\) را بیابید که به ازای آن‌ها، تابع \(f(x)\) در نقطه‌ی \(x=2\) پیوسته باشد.

\vspace{1cm}
\hrulefill
\vspace{1cm}
% ------------------------------------------------------------------
% سوال صد و هفتاد و دوم (سوال درخواستی شما)
% ------------------------------------------------------------------
\section*{سوال ۱۷۲: مارپیچ کسری}
دو تابع \(f(x) = x - [x]\) و \(g(x) = \left[\cos\left(\frac{\pi x}{2}\right)\right]\) را در نظر بگیرید. تعداد نقاط ناپیوستگی تابع مرکب \(y = (g \circ f)(x)\) در بازه‌ی \( [0, 10) \) کدام است؟

\vspace{1cm}
\hrulefill
\vspace{1cm}

% ------------------------------------------------------------------
% سوال صد و هفتاد و سوم (بسیار چالشی - شبح گسسته)
% ------------------------------------------------------------------
\section*{سوال ۱۷۳: شبح گسسته}
دو تابع \(f(x) = [x] + [-x]\) و \(g(x) = x^2 - x\) را در نظر بگیرید. مجموع مقادیر تابع مرکب \(h(x) = (g \circ f)(x)\) در نقاط \(x=0.5\) و \(x=5\) کدام است؟

\vspace{1cm}
\hrulefill
\vspace{1cm}

% ------------------------------------------------------------------
% سوال صد و هفتاد و چهارم (آخرالزمانی - پژواک دره)
% ------------------------------------------------------------------
\section*{سوال ۱۷۴: پژواک دره}
دو تابع \(f(x) = (x-2)^2\) و \(g(u) = 2[u] - u\) را در نظر بگیرید. تعداد نقاط ناپیوستگی تابع مرکب \(h(x) = (g \circ f)(x)\) در بازه‌ی \( [0, 5] \) کدام است؟

\vspace{1cm}
\hrulefill
\vspace{1cm}

% ------------------------------------------------------------------
% سوال صد و هفتاد و پنجم (پارادوکس تقارن - نسخه‌ی نهایی)
% ------------------------------------------------------------------
\section*{سوال ۱۷۵: پارادوکس تقارن}
دو تابع پیوسته‌ی \(g_1(x)\) و \(g_2(x)\) را در نظر بگیرید. تابع \(h(x)\) به صورت زیر تعریف شده است:
\begin{displaymath}
	h(x) = 
	\begin{cases}
		g_1(x) & ;  [x] + [1-x] \text{ زوج باشد} \\
		g_2(x) & ; [x] + [1-x] \text{ فرد باشد}
	\end{cases}
\end{displaymath}
اگر تابع \(h(x)\) بر روی تمام اعداد حقیقی پیوسته باشد، کدام گزینه همواره صحیح است؟
\begin{enumerate}[label=(\arabic*)]
	\item \(g_1(k) = g_2(k)\) برای تمام \(k \in \mathbb{Z}\)
	\item \(g_1(x) = g_2(x)\) برای تمام \(x \in \mathbb{R}\)
	\item \(g_1(1) = g_2(1)\)
	\item چنین توابعی وجود ندارند
\end{enumerate}

\vspace{1cm}
\hrulefill
\vspace{1cm}

% ------------------------------------------------------------------
% سوال صد و هفتاد و ششم (آستانه‌ی وارون - نسخه‌ی نهایی)
% ------------------------------------------------------------------
\section*{سوال ۱۷۶: آستانه‌ی وارون}
تابع \(f(x) = x^2\) را برای \(x \ge 0\) و وارون آن \(f^{-1}(x)\) را در نظر بگیرید. تابع \(h(x)\) به صورت زیر تعریف شده است.
\begin{displaymath}
	h(x) = 
	\begin{cases}
		a[f(x)] & ; [f(x)] > [x] \\
		b[x] & ; \text{در غیر این صورت}
	\end{cases}
\end{displaymath}
اگر تابع \(h(x)\) در نقطه‌ی \(x=\sqrt{2}\) پیوسته باشد، مقدار \( \frac{a}{b} \) کدام است؟

\vspace{1cm}
\hrulefill
\vspace{1cm}

% ------------------------------------------------------------------
% سوال صد و هفتاد و هفتم (هزارتوی تودرتو - نسخه‌ی نهایی)
% ------------------------------------------------------------------
\section*{سوال ۱۷۷: هزارتوی تودرتو}
تابع \(f(x) = x^2\) را در نظر بگیرید. تابع جدید \(h(x)\) به صورت \(h(x) = f\left([x] + \left[f^{-1}(x)\right]\right)\) تعریف شده است. این تابع در بازه‌ی \( (0, 100) \) در چند نقطه ناپیوسته است؟

\vspace{1cm}
\hrulefill
\vspace{1cm}

% ------------------------------------------------------------------
% سوال صد و هفتاد و هشتم (خنثی‌سازی بازتابی - نسخه‌ی نهایی)
% ------------------------------------------------------------------
\section*{سوال ۱۷۸: خنثی‌سازی بازتابی}
تابع \(f(x)\) به صورت زیر تعریف شده است.
\begin{displaymath}
	f(x) = 
	\begin{cases}
		x-a & ; x \ge a \\
		1 & ; x < a
	\end{cases}
\end{displaymath}
اگر تابع \( h(x) = f(x) \cdot f(6-x) \) بر روی تمام اعداد حقیقی پیوسته باشد، مقدار ثابت \(a\) کدام است؟

\vspace{1cm}
\hrulefill
\vspace{1cm}

% ------------------------------------------------------------------
% سوال صد و هفتاد و نهم (چاه جاذبه - نسخه‌ی نهایی)
% ------------------------------------------------------------------
\section*{سوال ۱۷۹: چاه جاذبه}
دو تابع \(f(x)\) و \(g(u)\) را در نظر بگیرید:
\begin{align*}
	f(x) &= \begin{cases} 2x-7 &; x \ge 2 \\ 2x+1 &; x < 2 \end{cases} \\
	g(u) &= u^2 - au + 1
\end{align*}
اگر تابع مرکب \( h(x) = (g \circ f)(x) \) در نقطه‌ی \(x=2\) پیوسته باشد، مقدار پارامتر \(a\) را بیابید.

\vspace{1cm}
\hrulefill
\vspace{1cm}

% ------------------------------------------------------------------
% سوال صد و هشتادم (کسر قابل تعمیر - نسخه‌ی نهایی)
% ------------------------------------------------------------------
\section*{سوال ۱۸۰: کسر قابل تعمیر}
دو تابع زیر را در نظر بگیرید:
\begin{align*}
	f(x) &= \begin{cases} x-a &; x \ge 3 \\ a-x &; x < 3 \end{cases} \\
	g(x) &= x^2-9
\end{align*}
اگر حد تابع \( h(x) = \frac{g(x)}{f(x)} \) در نقطه‌ی \(x=a\) موجود و متناهی باشد، مقدار ثابت \(a\) کدام است؟

\vspace{1cm}
\hrulefill
\vspace{1cm}

% ------------------------------------------------------------------
% سوال صد و هشتاد و یکم (پژواک‌های زنجیره‌ای - نسخه‌ی نهایی)
% ------------------------------------------------------------------
\section*{سوال ۱۸۱: پژواک‌های زنجیره‌ای}
تابع \(f(x)\) به صورت زیر تعریف شده است.
\begin{displaymath}
	f(x) = 
	\begin{cases}
		2x & ; 0 \le x < 1 \\
		4-x & ; 1 \le x \le 4
	\end{cases}
\end{displaymath}
تعداد نقاط ناپیوستگی تابع مرکب \( h(x) = (f \circ f)(x) \) در دامنه‌اش کدام است؟

\vspace{1cm}
\hrulefill
\vspace{1cm}

% ------------------------------------------------------------------
% سوال صد و هشتاد و دوم (سوال درخواستی شما)
% ------------------------------------------------------------------
\section*{سوال ۱۸۲: مرز پیوستگی}
تابع \( f(x) = [\log_5 x] + 2 \) در بازه‌ی \( (5, k^2+10) \) پیوسته است. کدام یک از مقادیر زیر برای \(k\) \textbf{نمی‌تواند} صحیح باشد؟
\begin{enumerate}[label=(\arabic*)]
	\item \( \pm 4 \)
	\item \( \pm \sqrt{14} \)
	\item \( \pm 3 \)
	\item \( \pm \sqrt{7} \)
\end{enumerate}

\vspace{1cm}
\hrulefill
\vspace{1cm}

% ------------------------------------------------------------------
% سوال صد و هشتاد و سوم (بسیار چالشی - وارون در مه)
% ------------------------------------------------------------------
% ------------------------------------------------------------------
% سوال صد و هشتاد و سوم (خنثی‌سازی معکوس - نسخه‌ی نهایی)
% ------------------------------------------------------------------
% ------------------------------------------------------------------
% سوال صد و هشتاد و سوم (تعادل وارون نزولی - نسخه‌ی قطعی)
% ------------------------------------------------------------------
\section*{سوال ۱۸۳: تعادل وارون نزولی}
تابع \(f(x) = \log_2(a-x)\) را در نظر بگیرید. تابع جدید \(h(x)\) به صورت زیر تعریف شده است:
\begin{displaymath}
	h(x) = [x] + [f^{-1}(x)]
\end{displaymath}
که در آن \(f^{-1}(x)\) تابع وارون \(f(x)\) است. اگر تابع \(h(x)\) در نقطه‌ی \(x=3\) پیوسته باشد و مقدار آن در این نقطه برابر با \(h(3)=12\) باشد، مقدار ثابت \(a\) کدام است؟

\end{document}